\let\negmedspace\undefined
\let\negthickspace\undefined
\documentclass[journal]{IEEEtran}
\usepackage[a5paper, margin=10mm, onecolumn]{geometry}
%\usepackage{lmodern} % Ensure lmodern is loaded for pdflatex
\usepackage{tfrupee} % Include tfrupee package

\setlength{\headheight}{1cm} % Set the height of the header box
\setlength{\headsep}{0mm}     % Set the distance between the header box and the top of the text

\usepackage{gvv-book}
\usepackage{gvv}
\usepackage{cite}
\usepackage{amsmath,amssymb,amsfonts,amsthm}
\usepackage{algorithmic}
\usepackage{graphicx}
\usepackage{textcomp}
\usepackage{xcolor}
%\usepackage{txfonts}
\usepackage{listings}
\usepackage{enumitem}
\usepackage{mathtools}
\usepackage{gensymb}
\usepackage{comment}
\usepackage[breaklinks=true]{hyperref}
\usepackage{tkz-euclide} 
\usepackage{listings}
% \usepackage{gvv}                                        
\def\inputGnumericTable{}                                 
\usepackage[latin1]{inputenc}                                
\usepackage{color}                                            
\usepackage{array}                                            
\usepackage{longtable}                                       
\usepackage{calc}                                             
\usepackage{multirow}                                         
\usepackage{hhline}                                           
\usepackage{ifthen}                                           
\usepackage{lscape}
\usepackage{circuitikz}
\tikzstyle{block} = [rectangle, draw, fill=blue!20, 
    text width=4em, text centered, rounded corners, minimum height=3em]
\tikzstyle{sum} = [draw, fill=blue!10, circle, minimum size=1cm, node distance=1.5cm]
\tikzstyle{input} = [coordinate]
\tikzstyle{output} = [coordinate]


\begin{document}

\bibliographystyle{IEEEtran}
\vspace{3cm}

\title{10.2.7}
\author{EE25BTECH11001 - Aarush Dilawri}
\maketitle
{\let\newpage\relax\maketitle}

\renewcommand{\thefigure}{\theenumi}
\renewcommand{\thetable}{\theenumi}
\setlength{\intextsep}{10pt} % Space between text and floats

\numberwithin{equation}{enumi}
\numberwithin{figure}{enumi}
\renewcommand{\thetable}{\theenumi}

\textbf{Question}:\\
At what points on the curve $x^2 + y^2 - 2x - 4y + 1 = 0$, the tangents are parallel to
the y-axis?

\textbf{Solution:}\\

\begin{align}
\text{The general equation of a conic is } 
\vec{x}^\top\vec{V}\vec{x} + 2\vec{u}^\top\vec{x} + f = 0
\end{align}

Comparing with $x^2 + y^2 - 2x - 4y + 1 = 0$ , we get
\begin{align}
\vec{V} = \myvec{1 & 0 \\ 0 & 1},\quad
\vec{u} = \myvec{-1 \\ -2},\quad
f = 1
\end{align}

The centre and radius of the circle are given by
\begin{align}
\vec{c} = -\vec{u} = \myvec{1 \\ 2},\quad
r = \sqrt{\norm{\vec{u}}^2 - f} 
= \sqrt{(-1)^2 + (-2)^2 - 1} 
= \sqrt{4} = 2
\end{align}

The points of contact of the tangent are given by
\begin{align}
\vec{q}_{i,j} = \pm r\frac{\vec{n}_j}{\norm{\vec{n}_j}} - \vec{u} \quad i,j = 1,2
\end{align}

Since the tangents are parallel to the y-axis, the normal vector is
\begin{align}
\vec{n} = \myvec{1 \\ 0}
\end{align}

Substituting in $\brak{0.4}$,
\begin{align}
\vec{q}_{1,1} 
&= \pm 2 \frac{\myvec{1 \\ 0}}{\norm{\myvec{1 \\ 0}}} - \myvec{-1 \\ -2}\\
&= \pm 2\myvec{1 \\ 0} + \myvec{1 \\ 2}
\end{align}

Therefore, the two points of contact are
\begin{align}
\vec{q}_1 &= \myvec{1+2 \\ 2+0} = \myvec{3 \\ 2}\\
\vec{q}_2 &= \myvec{1-2 \\ 2+0} = \myvec{-1 \\ 2}
\end{align}

Hence, the required points are  $\myvec{3 \\ 2}$ and $\myvec{-1 \\ 2}$

See Figure,
\begin{figure}[h!]
    \centering
    \includegraphics[height=0.5\textheight, keepaspectratio]{figs/fig.png}
    \label{figure_1}
\end{figure}
\end{document}