\documentclass{beamer}
\usepackage[utf8]{inputenc}

\usetheme{Madrid}
\usecolortheme{default}
\usepackage{amsmath,amssymb,amsfonts,amsthm}
\usepackage{txfonts}
\usepackage{tkz-euclide}
\usepackage{listings}
\usepackage{adjustbox}
\usepackage{array}
\usepackage{tabularx}
\usepackage{gvv}
\usepackage{lmodern}
\usepackage{circuitikz}
\usepackage{tikz}
\usepackage{graphicx}
\usepackage{mathtools} 
\setbeamertemplate{page number in head/foot}[totalframenumber]

\title{10.2.7}
\date{\today}
\author{EE25BTECH11001 - Aarush Dilawri}

\begin{document}

\frame{\titlepage}

\begin{frame}{Question}
\textbf{Question}:\\
At what points on the curve $x^2 + y^2 - 2x - 4y + 1 = 0$, the tangents are parallel to
the y-axis?
\end{frame}

\begin{frame}{Solution}
\textbf{Solution:}\\

\begin{align}
\text{The general equation of a conic is } 
\vec{x}^\top\vec{V}\vec{x} + 2\vec{u}^\top\vec{x} + f = 0
\end{align}

Comparing with $x^2 + y^2 - 2x - 4y + 1 = 0$ , we get
\begin{align}
\vec{V} = \myvec{1 & 0 \\ 0 & 1},\quad
\vec{u} = \myvec{-1 \\ -2},\quad
f = 1
\end{align}
\end{frame}

\begin{frame}{Solution}
The centre and radius of the circle are given by
\begin{align}
\vec{c} = -\vec{u} = \myvec{1 \\ 2},\quad
r = \sqrt{\norm{\vec{u}}^2 - f} 
= \sqrt{(-1)^2 + (-2)^2 - 1} 
= \sqrt{4} = 2
\end{align}

The points of contact of the tangent are given by
\begin{align}
\vec{q}_{i,j} = \pm r\frac{\vec{n}_j}{\norm{\vec{n}_j}} - \vec{u} \quad i,j = 1,2
\end{align}
\end{frame}

\begin{frame}{Solution}
Since the tangents are parallel to the y-axis, the normal vector is
\begin{align}
\vec{n} = \myvec{1 \\ 0}
\end{align}

Substituting in $\brak{0.4}$,
\begin{align}
\vec{q}_{1,1} 
&= \pm 2 \frac{\myvec{1 \\ 0}}{\norm{\myvec{1 \\ 0}}} - \myvec{-1 \\ -2}\\
&= \pm 2\myvec{1 \\ 0} + \myvec{1 \\ 2}
\end{align}
\end{frame}

\begin{frame}{Solution}
Therefore, the two points of contact are
\begin{align}
\vec{q}_1 &= \myvec{1+2 \\ 2+0} = \myvec{3 \\ 2}\\
\vec{q}_2 &= \myvec{1-2 \\ 2+0} = \myvec{-1 \\ 2}
\end{align}

Hence, the required points are  $\myvec{3 \\ 2}$ and $\myvec{-1 \\ 2}$
\end{frame}

\begin{frame}{Figure}
\begin{figure}[h!]
    \centering
    \includegraphics[height=0.5\textheight, keepaspectratio]{figs/fig.png}
    \label{figure_1}
\end{figure}
\end{frame}

\begin{frame}{Codes}
\url{https://github.com/AarushDilawri/ee1030-2025/tree/main/ee25btech11001/MATGEO/10.2.7/codes}
\end{frame}
\end{document}