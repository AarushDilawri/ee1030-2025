\let\negmedspace\undefined
\let\negthickspace\undefined
\documentclass[journal]{IEEEtran}
\usepackage[a5paper, margin=10mm, onecolumn]{geometry}
%\usepackage{lmodern} % Ensure lmodern is loaded for pdflatex
\usepackage{tfrupee} % Include tfrupee package

\setlength{\headheight}{1cm} % Set the height of the header box
\setlength{\headsep}{0mm}     % Set the distance between the header box and the top of the text

\usepackage{gvv-book}
\usepackage{gvv}
\usepackage{cite}
\usepackage{amsmath,amssymb,amsfonts,amsthm}
\usepackage{algorithmic}
\usepackage{graphicx}
\usepackage{textcomp}
\usepackage{xcolor}
\usepackage{txfonts}
\usepackage{listings}
\usepackage{enumitem}
\usepackage{mathtools}
\usepackage{gensymb}
\usepackage{comment}
\usepackage[breaklinks=true]{hyperref}
\usepackage{tkz-euclide} 
\usepackage{listings}
% \usepackage{gvv}                                        
\def\inputGnumericTable{}                                 
\usepackage[latin1]{inputenc}                                
\usepackage{color}                                            
\usepackage{array}                                            
\usepackage{longtable}                                       
\usepackage{calc}                                             
\usepackage{multirow}                                         
\usepackage{hhline}                                           
\usepackage{ifthen}                                           
\usepackage{lscape}
\usepackage{circuitikz}
\tikzstyle{block} = [rectangle, draw, fill=blue!20, 
    text width=4em, text centered, rounded corners, minimum height=3em]
\tikzstyle{sum} = [draw, fill=blue!10, circle, minimum size=1cm, node distance=1.5cm]
\tikzstyle{input} = [coordinate]
\tikzstyle{output} = [coordinate]


\begin{document}

\bibliographystyle{IEEEtran}
\vspace{3cm}

\title{5.7.14}
\author{EE25BTECH11001 - Aarush Dilawri}
\maketitle
% \newpage
% \bigskip
{\let\newpage\relax\maketitle}

\renewcommand{\thefigure}{\theenumi}
\renewcommand{\thetable}{\theenumi}
\setlength{\intextsep}{10pt} % Space between text and floats


\numberwithin{equation}{enumi}
\numberwithin{figure}{enumi}
\renewcommand{\thetable}{\theenumi}

\textbf{Question}:\\
If $\vec{A} = \myvec{-3 & 6 \\ -2 & 4}$, then show that $\vec{A}^3 = \vec{A}$.

\solution \\

The characteristic equation of $\vec{A}$ is given by:
\begin{align}
    f\brak{\lambda} = \lvert\vec{A}-\lambda\vec{I}\rvert =0
\end{align}
Therefore,
\begin{align}
    f\brak{\lambda} = \mydet{-3-\lambda & 6 \\ -2 & 4-\lambda} =0\\
    f\brak{\lambda}=\lambda^2 -\lambda =0    
\end{align}
By Cayley-Hamilton theorem,
\begin{align}
    f\brak{\lambda} = f\brak{\vec{A}} =0
\end{align}
Therefore,
\begin{align}
    \vec{A}^2 - \vec{A} =0
    \implies \vec{A}^2 = \vec{A}
\end{align}
Pre-multiplying both sides by $\vec{A}$,
\begin{align}
    \vec{A}^3 &= \vec{A}^2\quad\text{but $\vec{A}^2 = \vec{A}$}\\
    \implies \vec{A}^3 = \vec{A}
\end{align}
Hence proved.
\newpage
See Figure,
\begin{figure}[h!]
    \centering
    \includegraphics[height=0.5\textheight, keepaspectratio]{figs/fig.png}
    \label{figure_1}
\end{figure}
\end{document}