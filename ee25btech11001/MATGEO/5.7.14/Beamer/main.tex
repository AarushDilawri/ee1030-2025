\documentclass{beamer}
\usepackage[utf8]{inputenc}

\usetheme{Madrid}
\usecolortheme{default}
\usepackage{amsmath,amssymb,amsfonts,amsthm}
\usepackage{txfonts}
\usepackage{tkz-euclide}
\usepackage{listings}
\usepackage{adjustbox}
\usepackage{array}
\usepackage{tabularx}
\usepackage{gvv}
\usepackage{lmodern}
\usepackage{circuitikz}
\usepackage{tikz}
\usepackage{graphicx}

\setbeamertemplate{page number in head/foot}[totalframenumber]

\title{5.7.14}
\date{\today}
\author{EE25BTECH11001 - Aarush Dilawri}

\begin{document}

\frame{\titlepage}

\begin{frame}{Question}
\textbf{Question}:\\
If $\vec{A} = \myvec{-3 & 6 \\ -2 & 4}$, then show that $\vec{A}^3 = \vec{A}$.
\end{frame}

\begin{frame}{Characteristic Polynomial}

The characteristic equation of $\vec{A}$ is given by:
\begin{align}
    f\brak{\lambda} = \lvert\vec{A}-\lambda\vec{I}\rvert =0
\end{align}
Therefore,
\begin{align}
    f\brak{\lambda} = \mydet{-3-\lambda & 6 \\ -2 & 4-\lambda} =0\\
    f\brak{\lambda}=\lambda^2 -\lambda =0    
\end{align}
\end{frame}

\begin{frame}{Applying Cayley-Hamilton}
By Cayley-Hamilton theorem,
\begin{align}
    f\brak{\lambda} = f\brak{\vec{A}} =0
\end{align}
Therefore,
\begin{align}
    \vec{A}^2 - \vec{A} =0
    \implies \vec{A}^2 = \vec{A}
\end{align}
\end{frame}

\begin{frame}{Final Step}
Pre-multiplying both sides by $\vec{A}$,
\begin{align}
    \vec{A}^3 &= \vec{A}^2\quad\text{but $\vec{A}^2 = \vec{A}$}\\
    \implies \vec{A}^3 = \vec{A}
\end{align}
Hence proved.
\end{frame}

\begin{frame}{Graphical Representation}
See Figure,
\begin{figure}[h!]
    \centering
    \includegraphics[height=0.5\textheight, keepaspectratio]{figs/fig.png}
    \label{figure_1}
\end{figure}
\end{frame}


\begin{frame}[fragile]{C Code (code.c)}
\begin{lstlisting}[language=C]
#include <stdio.h>

// Function to compute characteristic polynomial coefficients of a 2x2 matrix
// Input: a11, a12, a21, a22
// Output: coeffs[0] = 1 (λ^2), coeffs[1] = -trace(A), coeffs[2] = det(A)
void char_poly(double a11, double a12, double a21, double a22, double* coeffs) {
    double trace = a11 + a22;
    double det = a11 * a22 - a12 * a21;

    coeffs[0] = 1.0;       // λ^2 coefficient
    coeffs[1] = -trace;    // λ coefficient
    coeffs[2] = det;       // constant term
}
\end{lstlisting}
\end{frame}

\begin{frame}[fragile]{Python Code (code.py)}
\begin{lstlisting}[language=Python]
import numpy as np
import matplotlib.pyplot as plt

# Given matrix A
a11, a12, a21, a22 = -3, 6, -2, 4

# Compute trace and determinant
trace = a11 + a22
det = a11*a22 - a12*a21

# Polynomial coefficients
coeffs = [1, -trace, det]
print("Characteristic Polynomial Coefficients:", coeffs)

\end{lstlisting}
\end{frame}

\begin{frame}[fragile]{Python Code (code.py)}
\begin{lstlisting}[language=Python]
# Define polynomial
lam = np.linspace(-10, 10, 400)
poly_vals = coeffs[0]*lam**2 + coeffs[1]*lam + coeffs[2]

# Plot
plt.axhline(0, color='black', linewidth=0.8)
plt.plot(lam, poly_vals, label="Characteristic Polynomial")
plt.xlabel("λ")
plt.ylabel("p(λ)")
plt.legend()
plt.grid(True)
plt.show()
\end{lstlisting}
\end{frame}


\begin{frame}[fragile]{Python Code (nativecode.py)}
\begin{lstlisting}[language=Python]
import ctypes
import numpy as np
import matplotlib.pyplot as plt

# Load the shared library
code = ctypes.CDLL("./code.so")

# Define argument and return types
code.char_poly.argtypes = [ctypes.c_double, ctypes.c_double, ctypes.c_double, ctypes.c_double,
                           np.ctypeslib.ndpointer(dtype=np.float64, ndim=1, flags="C")]

coeffs = np.zeros(3, dtype=np.float64)

a11, a12, a21, a22 = -3, 6, -2, 4
\end{lstlisting}
\end{frame}
\begin{frame}[fragile]{Python Code (nativecode.py)}
\begin{lstlisting}[language=Python]
# Call C function
code.char_poly(a11, a12, a21, a22, coeffs)
print("Characteristic Polynomial Coefficients:", coeffs)

# Define polynomial
lam = np.linspace(-10, 10, 400)
poly_vals = coeffs[0]*lam**2 + coeffs[1]*lam + coeffs[2]

# Plot
plt.axhline(0, color='black', linewidth=0.8)
plt.plot(lam, poly_vals, label="Characteristic Polynomial")
plt.xlabel("λ")
plt.ylabel("p(λ)")
plt.legend()
plt.grid(True)
plt.show()

\end{lstlisting}
\end{frame}

\end{document}