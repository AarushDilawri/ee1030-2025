\documentclass{beamer}
\usepackage[utf8]{inputenc}

\usetheme{Madrid}
\usecolortheme{default}
\usepackage{amsmath,amssymb,amsfonts,amsthm}
\usepackage{txfonts}
\usepackage{tkz-euclide}
\usepackage{listings}
\usepackage{adjustbox}
\usepackage{array}
\usepackage{tabularx}
\usepackage{gvv}
\usepackage{lmodern}
\usepackage{circuitikz}
\usepackage{tikz}
\usepackage{graphicx}
\usepackage{mathtools} 
\setbeamertemplate{page number in head/foot}[totalframenumber]

\title{5.3.35}
\date{\today}
\author{EE25BTECH11001 - Aarush Dilawri}

\begin{document}

\frame{\titlepage}
\begin{frame}{Question}
\textbf{Question}:\\
If the pair of equations
\begin{align}
    3x-y+8=0\\
    6x-ry+16=0
\end{align}
represent coincident lines, then find the value of $r$.
\end{frame}

\begin{frame}{Solution}

The equation of line:
\begin{align}
\vec{n}^\top\vec{x}=c
\end{align}

Line L:
\begin{align}
\myvec{3 & -1}\myvec{x \\ y}=-8
\end{align}

Line K:
\begin{align}
\myvec{6 & -r}\myvec{x \\ y}=-16
\end{align}
\end{frame}

\begin{frame}{Matrix Form}
These can be combined and written in matrix form:
\begin{align}
\myvec{3 & -1 \\ 6 & -r}\myvec{x \\ y} = \myvec{-8 \\ -16}
\end{align}

The following augmented matrix can be solved by gaussian elimination
\begin{align}
\augvec{2}{1}{3 & -1 & -8 \\ 6 & -r & -16} 
\xleftrightarrow{R_2 \rightarrow R_2 - 2R_1} 
\augvec{2}{1}{3 & -1 & -8 \\ 0 & -r+2 & 0}
\end{align}
\end{frame}

\begin{frame}{Rouché–Capelli Application}
Since the lines are coincident, they have infinitely many solutions.
\begin{align}
    \text{Thus, rank}\brak{\vec{A}} = \text{rank}\brak{[\vec{A} \mid b]}<n \quad\text{where $n$ is the number of variables.}
\end{align}
\begin{align}
    \implies -r+2 =0
    \implies r=2
\end{align}
Hence, the value of $r$ is $2$.
\end{frame}

\begin{frame}{Graphical Representation}
See Figure,
\begin{figure}[h!]
    \centering
    \includegraphics[height=0.5\textheight, keepaspectratio]{figs/fig.png}
    \label{figure_1}
\end{figure}
\end{frame}
\begin{frame}[fragile]{C Code (code.c)}
\begin{lstlisting}[language=C]
#include <stdio.h>

// Function to find r such that the lines are coincident
// Lines: a1*x + b1*y + c1 = 0
//        a2*x + (-r)*y + c2 = 0
double find_r(double a1, double b1, double c1, double a2, double c2) {
    // Since line2 normal = (a2, -r), it must be proportional to (a1, b1)
    // So, a2/a1 = (-r)/b1  AND  c2/c1 = a2/a1
    double k = a2 / a1; 
    double r = -k * b1; 
    return r;
}


\end{lstlisting}
\end{frame}

\begin{frame}[fragile]{Python Code (code.py)}
\begin{lstlisting}[language=Python]
import numpy as np
import matplotlib.pyplot as plt

a1, b1, c1 = 3, -1, 8
a2, b2, c2 = 6, -2, 16   # since r = 2

def line1(x): return (a1*x + c1)/(-b1)
def line2(x): return (a2*x + c2)/(-b2)

x = np.linspace(-10, 10, 100)
plt.plot(x, line1(x), label="Line 1")
plt.plot(x, line2(x), label="Line 2 (r=2)")
plt.legend()
plt.grid(True)
plt.show()

\end{lstlisting}
\end{frame}

\begin{frame}[fragile]{Python Code (nativecode.py)}
\begin{lstlisting}[language=Python]
import ctypes
import numpy as np
import matplotlib.pyplot as plt

# Load the shared library
code = ctypes.CDLL("./code.so")

# Declare function signature
code.find_r.restype = ctypes.c_double
code.find_r.argtypes = [ctypes.c_double, ctypes.c_double, ctypes.c_double, ctypes.c_double, ctypes.c_double]

# Given coefficients
a1, b1, c1 = 3, -1, 8
a2, c2 = 6, 16




\end{lstlisting}
\end{frame}

\begin{frame}[fragile]{Python Code (nativecode.py)}
\begin{lstlisting}[language=Python]
# Call C function
r = code.find_r(a1, b1, c1, a2, c2)
print("Value of r:", r)

# Define line equations
def line1(x): return (a1*x + c1)/(-b1)
def line2(x): return (a2*x + c2)/(-(-r))

# Plot
x = np.linspace(-10, 10, 100)
plt.plot(x, line1(x), label="Line 1")
plt.plot(x, line2(x), label="Line 2 (with r)")
plt.legend()
plt.grid(True)
plt.show()

\end{lstlisting}
\end{frame}



\end{document}
