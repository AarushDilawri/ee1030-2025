\let\negmedspace\undefined
\let\negthickspace\undefined
\documentclass[journal]{IEEEtran}
\usepackage[a5paper, margin=10mm, onecolumn]{geometry}
%\usepackage{lmodern} % Ensure lmodern is loaded for pdflatex
\usepackage{tfrupee} % Include tfrupee package

\setlength{\headheight}{1cm} % Set the height of the header box
\setlength{\headsep}{0mm}     % Set the distance between the header box and the top of the text

\usepackage{gvv-book}
\usepackage{gvv}
\usepackage{cite}
\usepackage{amsmath,amssymb,amsfonts,amsthm}
\usepackage{algorithmic}
\usepackage{graphicx}
\usepackage{textcomp}
\usepackage{xcolor}
%\usepackage{txfonts}
\usepackage{listings}
\usepackage{enumitem}
\usepackage{mathtools}
\usepackage{gensymb}
\usepackage{comment}
\usepackage[breaklinks=true]{hyperref}
\usepackage{tkz-euclide} 
\usepackage{listings}
% \usepackage{gvv}                                        
\def\inputGnumericTable{}                                 
\usepackage[latin1]{inputenc}                                
\usepackage{color}                                            
\usepackage{array}                                            
\usepackage{longtable}                                       
\usepackage{calc}                                             
\usepackage{multirow}                                         
\usepackage{hhline}                                           
\usepackage{ifthen}                                           
\usepackage{lscape}
\usepackage{circuitikz}
\tikzstyle{block} = [rectangle, draw, fill=blue!20, 
    text width=4em, text centered, rounded corners, minimum height=3em]
\tikzstyle{sum} = [draw, fill=blue!10, circle, minimum size=1cm, node distance=1.5cm]
\tikzstyle{input} = [coordinate]
\tikzstyle{output} = [coordinate]


\begin{document}

\bibliographystyle{IEEEtran}
\vspace{3cm}

\title{5.3.35}
\author{EE25BTECH11001 - Aarush Dilawri}
\maketitle
{\let\newpage\relax\maketitle}

\renewcommand{\thefigure}{\theenumi}
\renewcommand{\thetable}{\theenumi}
\setlength{\intextsep}{10pt} % Space between text and floats

\numberwithin{equation}{enumi}
\numberwithin{figure}{enumi}
\renewcommand{\thetable}{\theenumi}

\textbf{Question}:\\
If the pair of equations\\
\begin{align}
    3x-y+8=0\\
    6x-ry+16=0
\end{align}
represent coincident lines, then find the value of $r$.\\
\solution\\
The equation of line:

\begin{align}
\vec{n}^\top\vec{x}=c
\end{align}

Line L:
\begin{align}
\myvec{3 & -1}\myvec{x \\ y}=-8
\end{align}

Line K:
\begin{align}
\myvec{6 & -r}\myvec{x \\ y}=-16
\end{align}

These can be combined and written in matrix form:
\begin{align}
\myvec{3 & -1 \\ 6 & -r}\myvec{x \\ y} = \myvec{-8 \\ -16}
\end{align}

The following augmented matrix can be solved by gaussian elimination
\begin{align}
\augvec{2}{1}{3 & -1 & -8 \\ 6 & -r & -16} \xleftrightarrow{R_2 \rightarrow R_2 - 2R_1} \augvec{2}{1}{3 & -1 & -8 \\ 0 & -r+2 & 0}
\end{align}
Since the lines are coincident, they have infinitely many solutions.
\begin{align}
    \text{Thus, rank}\brak{\vec{A}} = \text{rank}\brak{[\vec{A} \mid b]}<n \quad\text{where $n$ is the number of variables.}
\end{align}
\begin{align}
    \implies -r+2 =0
    \implies r=2
\end{align}
Hence, the value of $r$ is $2$.
\newpage
See Figure,
\begin{figure}[h!]
    \centering
    \includegraphics[height=0.5\textheight, keepaspectratio]{figs/fig.png}
    \label{figure_1}
\end{figure}
\end{document}
