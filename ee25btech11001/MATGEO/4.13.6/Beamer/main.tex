\documentclass{beamer}
\usepackage[utf8]{inputenc}

\usetheme{Madrid}
\usecolortheme{default}
\usepackage{amsmath,amssymb,amsfonts,amsthm}
\usepackage{txfonts}
\usepackage{tkz-euclide}
\usepackage{listings}
\usepackage{adjustbox}
\usepackage{array}
\usepackage{tabularx}
\usepackage{gvv}
\usepackage{lmodern}
\usepackage{circuitikz}
\usepackage{tikz}
\usepackage{graphicx}
\usepackage{mathtools} 
\setbeamertemplate{page number in head/foot}[totalframenumber]

\title{4.13.6}
\date{\today}
\author{EE25BTECH11001 - Aarush Dilawri}

\begin{document}

\frame{\titlepage}

\begin{frame}{Problem Statement}
\textbf{Question}:\\
Given the points $\vec{A} \brak{0,4}$ and $\vec{B} \brak{0,-4}$, the equation of the locus of the point $\vec{p}\brak{x,y}$, such that $\lvert AP -BP \rvert = 6$
\end{frame}

\begin{frame}{Step 1: Define Vectors}
\begin{align}
    \vec{A},\,\vec{B},\,\vec{P} &\in \mathbb{R}^n
\end{align}

Let the given scalar be $ \delta \ge 0 $ and choose the sign $s\in\{+1,-1\}$ so that
\begin{align}
    r_1 - r_2 &= s\,\delta,
\end{align}
where $r_1=\|\vec{P}-\vec{A}\|$ and $r_2=\|\vec{P}-\vec{B}\|$.
\end{frame}

\begin{frame}{Step 2: Difference Vector}
Define the difference vector
\begin{align}
    \vec{u} &= \vec{A}-\vec{B},
\end{align}
and the shorthand
\begin{align}
    D &= s\,\delta, \qquad
    \alpha \;=\; \vec{A}^\top\vec{A} - \vec{B}^\top\vec{B}.
\end{align}
\end{frame}

\begin{frame}{Step 3: Difference of Squares}
\begin{align}
    \|\vec{P}-\vec{A}\|^2 - \|\vec{P}-\vec{B}\|^2
    &= (\vec{P}-\vec{A})^\top(\vec{P}-\vec{A}) - (\vec{P}-\vec{B})^\top(\vec{P}-\vec{B}) \\
    &= -2\,\vec{P}^\top\vec{u} + \vec{A}^\top\vec{A} - \vec{B}^\top\vec{B}
    \;=\; -2\,\vec{P}^\top\vec{u} + \alpha .
\end{align}
\end{frame}

\begin{frame}{Step 4: Using $r_1^2 - r_2^2$ Identity}
Use $(r_1-r_2)(r_1+r_2)=r_1^2-r_2^2$ and $r_1-r_2 = D$ to get
\begin{align}
    D\,(r_1+r_2) &= -2\,\vec{P}^\top\vec{u} + \alpha
    \quad\Longrightarrow\quad
    r_1+r_2 \;=\; \frac{-2\,\vec{P}^\top\vec{u} + \alpha}{D}.
\end{align}

Hence
\begin{align}
    r_1 &= \frac{(r_1-r_2)+(r_1+r_2)}{2}
        \;=\; \frac{D}{2} \;+\; \frac{\alpha}{2D} \;-\; \frac{\vec{P}^\top\vec{u}}{D}.
\end{align}
\end{frame}

\begin{frame}{Step 5: Quadratic Form}
Square this expression and equate to the explicit quadratic form for $r_1^2$:
\begin{align}
    \left(\frac{D}{2} + \frac{\alpha}{2D} - \frac{\vec{P}^\top\vec{u}}{D}\right)^2
    &= (\vec{P}-\vec{A})^\top(\vec{P}-\vec{A})
    \;=\; \vec{P}^\top\vec{P} - 2\,\vec{P}^\top\vec{A} + \vec{A}^\top\vec{A}.
\end{align}

Multiply both sides by $D^2$ and simplify to get the general quadratic (conic) equation:
\begin{align}
    \vec{P}^\top\big(\vec{u}\vec{u}^\top - D^2 I\big)\vec{P}
    \;+\;
    \big(-(D^2+\alpha)\,\vec{u} + 2D^2\,\vec{A}\big)^\top \vec{P}
    \;+\;
    \frac{(D^2+\alpha)^2}{4} - D^2\,\vec{A}^\top\vec{A}
    \;=\; 0.
\end{align}
\end{frame}

\begin{frame}{Step 6: Substitute Values}
Substitute $\vec{A}=\myvec{0\\4}$, $\vec{B}=\myvec{0\\-4}$ and $\delta=6$.
\begin{align}
    \vec{u} &= \vec{A}-\vec{B} = \myvec{0\\8} \\
    \alpha &= \vec{A}^\top\vec{A} - \vec{B}^\top\vec{B} = 16 - 16 = 0 \\
    D &= s\,\delta = \pm 6 \quad\Rightarrow\quad D^2 = 36
\end{align}
\end{frame}

\begin{frame}{Step 7: Quadratic Matrix Equation}
\begin{align}
    \vec{P}^\top\big(\vec{u}\vec{u}^\top - 36I\big)\vec{P}
    + \big(-D^2\vec{u} + 2D^2\vec{A}\big)^\top \vec{P}
    + \frac{D^4}{4} - 36\vec{A}^\top\vec{A} &= 0
\end{align}

Compute each term:
\begin{align}
    \vec{u}\vec{u}^\top &= \myvec{0 & 0 \\ 0 & 64}, \qquad
    I = \myvec{1 & 0 \\ 0 & 1} \\
    \vec{u}\vec{u}^\top - 36I &= \myvec{-36 & 0 \\ 0 & 28}
\end{align}
\end{frame}

\begin{frame}{Step 8: Linear and Constant Terms}
Linear coefficient:
\begin{align}
    -D^2\vec{u} + 2D^2\vec{A} &= -36\myvec{0\\8} + 72\myvec{0\\4} = \myvec{0\\0}
\end{align}

Constant term:
\begin{align}
    \frac{D^4}{4} - 36\vec{A}^\top\vec{A} = 324 - 576 = -252
\end{align}
\end{frame}

\begin{frame}{Step 9: Locus Equation}
The locus is
\begin{align}
    \vec{P}^\top\myvec{-36 & 0 \\ 0 & 28}\vec{P} - 252 &= 0
    \quad\Longrightarrow\quad
    -36x^2 + 28y^2 = 252
\end{align}

Dividing through:
\begin{align}
    \frac{y^2}{9} - \frac{x^2}{7} &= 1
\end{align}

Thus, the locus is a hyperbola centered at the origin:
\begin{align}
    \frac{y^2}{9} - \frac{x^2}{7} = 1
\end{align}
\end{frame}
\begin{frame}{Figure}
\begin{figure}[h!]
    \centering
    \includegraphics[height=0.5\textheight, keepaspectratio]{figs/fig.png}
    \label{figure_1}
\end{figure}
\end{frame}

\begin{frame}[fragile]{C Code (code.c)}
\begin{lstlisting}[language=C]
#include <stdio.h>
#include <math.h>
double inner_product(int n, double *u, double *v) {
    double sum = 0.0;
    for(int i=0; i<n; i++) {
        sum += u[i]*v[i];
    }
    return sum;
}
double locus_value(int n, double *A, double *B, double *P, double D) {
    // Compute u = A - B
    double u[10];  // assume dimension <= 10 for simplicity
    for(int i=0; i<n; i++) {
        u[i] = A[i] - B[i];
    }
\end{lstlisting}
\end{frame}
\begin{frame}[fragile]{C Code (code.c)}
\begin{lstlisting}[language=C]
    double alpha = inner_product(n, A, A) - inner_product(n, B, B);
    double uuT_P[n];
    for(int i=0; i<n; i++) {
        uuT_P[i] = u[i]*inner_product(n, u, P);
    }
    double quad = inner_product(n, P, uuT_P);
    quad -= D*D * inner_product(n, P, P);
    double coeff[n];
    for(int i=0; i<n; i++) {
        coeff[i] = -(D*D+alpha)*u[i] + 2*D*D*A[i];
    }
    double lin = inner_product(n, coeff, P);
    double constant = ((D*D+alpha)*(D*D+alpha))/4.0 - D*D*inner_product(n, A, A);

    return quad + lin + constant;
}
\end{lstlisting}
\end{frame}

\begin{frame}[fragile]{Python Code (code.py)}
\begin{lstlisting}[language=Python]
import numpy as np
import matplotlib.pyplot as plt

def inner_product(u, v):
    return np.dot(u, v)

def locus_value(A, B, P, D):
    u = A - B
    alpha = inner_product(A, A) - inner_product(B, B)
    quad = (np.dot(P, u))**2 - D*D*inner_product(P, P)
    coeff = -(D*D+alpha)*u + 2*D*D*A
    lin = inner_product(coeff, P)
    constant = ((D*D+alpha)**2)/4.0 - D*D*inner_product(A, A)

    return quad + lin + constant

\end{lstlisting}
\end{frame}

\begin{frame}[fragile]{Python Code (code.py)}
\begin{lstlisting}[language=Python]
# Example: A=(0,4), B=(0,-4), delta=6
A = np.array([0.0, 4.0])
B = np.array([0.0, -4.0])
D = 6.0

x = np.linspace(-10, 10, 400)
y = np.linspace(-10, 10, 400)
X, Y = np.meshgrid(x, y)
Z = np.zeros_like(X)



\end{lstlisting}
\end{frame}


\begin{frame}[fragile]{Python Code (code.py)}
\begin{lstlisting}[language=Python]

for i in range(X.shape[0]):
    for j in range(X.shape[1]):
        P = np.array([X[i,j], Y[i,j]])
        Z[i,j] = locus_value(A, B, P, D)

plt.contour(X, Y, Z, levels=[0], colors="red")
plt.axhline(0, color="k", linewidth=0.5)
plt.axvline(0, color="k", linewidth=0.5)
plt.gca().set_aspect("equal")
plt.title("Locus using pure Python")
plt.show()
\end{lstlisting}
\end{frame}
\begin{frame}[fragile]{Python Code (nativecode.py)}
\begin{lstlisting}[language=Python]
import ctypes
import numpy as np
import matplotlib.pyplot as plt

# Load the shared library
lib = ctypes.CDLL("./code.so")

# Define function signature
lib.locus_value.argtypes = [
    ctypes.c_int,                           # dimension
    np.ctypeslib.ndpointer(dtype=np.double), # A
    np.ctypeslib.ndpointer(dtype=np.double), # B
    np.ctypeslib.ndpointer(dtype=np.double), # P
    ctypes.c_double                         # D
]
lib.locus_value.restype = ctypes.c_double
\end{lstlisting}
\end{frame}
\begin{frame}[fragile]{Python Code (nativecode.py)}
\begin{lstlisting}[language=Python]
# Example: A=(0,4), B=(0,-4), delta=6
A = np.array([0.0, 4.0], dtype=np.double)
B = np.array([0.0, -4.0], dtype=np.double)
D = 6.0

# Create grid and evaluate locus
x = np.linspace(-10, 10, 400)
y = np.linspace(-10, 10, 400)
X, Y = np.meshgrid(x, y)
Z = np.zeros_like(X)
\end{lstlisting}
\end{frame}

\begin{frame}[fragile]{Python Code (nativecode.py)}
\begin{lstlisting}[language=Python]
for i in range(X.shape[0]):
    for j in range(X.shape[1]):
        P = np.array([X[i,j], Y[i,j]], dtype=np.double)
        Z[i,j] = lib.locus_value(2, A, B, P, D)

# Plot contour Z=0 (the locus)
plt.contour(X, Y, Z, levels=[0], colors="blue")
plt.axhline(0, color="k", linewidth=0.5)
plt.axvline(0, color="k", linewidth=0.5)
plt.gca().set_aspect("equal")
plt.title("Locus using C library")
plt.show()
\end{lstlisting}
\end{frame}

\end{document}

