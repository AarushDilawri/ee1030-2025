\let\negmedspace\undefined
\let\negthickspace\undefined
\documentclass[journal]{IEEEtran}
\usepackage[a5paper, margin=10mm, onecolumn]{geometry}
%\usepackage{lmodern} % Ensure lmodern is loaded for pdflatex
\usepackage{tfrupee} % Include tfrupee package

\setlength{\headheight}{1cm} % Set the height of the header box
\setlength{\headsep}{0mm}     % Set the distance between the header box and the top of the text

\usepackage{gvv-book}
\usepackage{gvv}
\usepackage{cite}
\usepackage{amsmath,amssymb,amsfonts,amsthm}
\usepackage{algorithmic}
\usepackage{graphicx}
\usepackage{textcomp}
\usepackage{xcolor}
\usepackage{txfonts}
\usepackage{listings}
\usepackage{enumitem}
\usepackage{mathtools}
\usepackage{gensymb}
\usepackage{comment}
\usepackage[breaklinks=true]{hyperref}
\usepackage{tkz-euclide} 
\usepackage{listings}
% \usepackage{gvv}                                        
\def\inputGnumericTable{}                                 
\usepackage[latin1]{inputenc}                                
\usepackage{color}                                            
\usepackage{array}                                            
\usepackage{longtable}                                       
\usepackage{calc}                                             
\usepackage{multirow}                                         
\usepackage{hhline}                                           
\usepackage{ifthen}                                           
\usepackage{lscape}
\usepackage{circuitikz}
\tikzstyle{block} = [rectangle, draw, fill=blue!20, 
    text width=4em, text centered, rounded corners, minimum height=3em]
\tikzstyle{sum} = [draw, fill=blue!10, circle, minimum size=1cm, node distance=1.5cm]
\tikzstyle{input} = [coordinate]
\tikzstyle{output} = [coordinate]


\begin{document}

\bibliographystyle{IEEEtran}
\vspace{3cm}

\title{4.13.6}
\author{EE25BTECH11001 - Aarush Dilawri}
\maketitle
% \newpage
% \bigskip
{\let\newpage\relax\maketitle}

\renewcommand{\thefigure}{\theenumi}
\renewcommand{\thetable}{\theenumi}
\setlength{\intextsep}{10pt} % Space between text and floats


\numberwithin{equation}{enumi}
\numberwithin{figure}{enumi}
\renewcommand{\thetable}{\theenumi}

\textbf{Question}:\\
Given the points $\vec{A} \brak{0,4}$ and $\vec{B} \brak{0,-4}$, the equation of the locus of the point $\vec{p}\brak{x,y}$, such that $\brak{AP - BP}^2 = 6^2$

\solution \\
\begin{align}
    \vec{A},\,\vec{B},\,\vec{P} &\in \mathbb{R}^n
\end{align}

Let the given scalar be $ \delta \ge 0 $ 
\begin{align}
    \brak{r_1 -r_2}^2 = \delta^2
\end{align}
where $r_1=\|\vec{P}-\vec{A}\|$ and $r_2=\|\vec{P}-\vec{B}\|$.\\

Taking square root, 
\begin{align}
    \lvert r_1 - r_2 \rvert = \pm\delta
\end{align}
Let's define
\begin{align}
    D = s\,\delta
\end{align}
where $s\in \cbrak{+1,-1}$ such that
\begin{align}
    r_1 -r_2 = s\delta = D
\end{align}
Let's find $r_1^2 -r_2^2$
\begin{align}
    \|\vec{P}-\vec{A}\|^2 - \|\vec{P}-\vec{B}\|^2
    &= (\vec{P}-\vec{A})^\top(\vec{P}-\vec{A}) - (\vec{P}-\vec{B})^\top(\vec{P}-\vec{B}) \\
    &= -2\,\vec{P}^\top\vec{u} + \vec{A}^\top\vec{A} - \vec{B}^\top\vec{B}
    \;=\; -2\,\vec{P}^\top\vec{u} + \alpha .
\end{align}
where
\begin{align}
    \vec{u} = \vec{A} -\vec{B} \quad \text{and} \quad  \alpha \;=\; \vec{A}^\top\vec{A} - \vec{B}^\top\vec{B}.
\end{align}

Use $(r_1-r_2)(r_1+r_2)=r_1^2-r_2^2$ and $r_1-r_2 = D$ to get
\begin{align}
    D\,(r_1+r_2) &= -2\,\vec{P}^\top\vec{u} + \alpha
    \quad\Longrightarrow\quad
    r_1+r_2 \;=\; \frac{-2\,\vec{P}^\top\vec{u} + \alpha}{D}.
\end{align}

Hence
\begin{align}
    r_1 &= \frac{(r_1-r_2)+(r_1+r_2)}{2}
        \;=\; \frac{D}{2} \;+\; \frac{\alpha}{2D} \;-\; \frac{\vec{P}^\top\vec{u}}{D}.
\end{align}

Square this expression and equate to the explicit quadratic form for $r_1^2$:
\begin{align}
    \left(\frac{D}{2} + \frac{\alpha}{2D} - \frac{\vec{P}^\top\vec{u}}{D}\right)^2
    &= (\vec{P}-\vec{A})^\top(\vec{P}-\vec{A})
    \;=\; \vec{P}^\top\vec{P} - 2\,\vec{P}^\top\vec{A} + \vec{A}^\top\vec{A}.
\end{align}

Multiply both sides by $D^2$ and simplify.  After collecting terms one obtains the general quadratic (conic) equation in the vector \(\vec{P}\):
\begin{align}
    \vec{P}^\top\big(\vec{u}\vec{u}^\top - D^2 I\big)\vec{P}
    \;+\;
    \big(-(D^2+\alpha)\,\vec{u} + 2D^2\,\vec{A}\big)^\top \vec{P}
    \;+\;
    \frac{(D^2+\alpha)^2}{4} - D^2\,\vec{A}^\top\vec{A}
    \;=\; 0.
\end{align}
Now substitute $\vec{A}=\myvec{0\\4}$, $\vec{B}=\myvec{0\\-4}$ and $\delta=6$.

\begin{align}
    \vec{u} &= \vec{A}-\vec{B} = \myvec{0\\8} \\
    \alpha &= \vec{A}^\top\vec{A} - \vec{B}^\top\vec{B} = 16 - 16 = 0 \\
    D &= s\,\delta = \pm 6 \quad\Rightarrow\quad D^2 = 36
\end{align}

The quadratic matrix equation becomes
\begin{align}
    \vec{P}^\top\big(\vec{u}\vec{u}^\top - 36I\big)\vec{P}
    + \big(-D^2\vec{u} + 2D^2\vec{A}\big)^\top \vec{P}
    + \frac{D^4}{4} - 36\vec{A}^\top\vec{A} &= 0
\end{align}

Now compute each term.

\begin{align}
    \vec{u}\vec{u}^\top &= \myvec{0 & 0 \\ 0 & 64}, \qquad
    I = \myvec{1 & 0 \\ 0 & 1}
\end{align}

So
\begin{align}
    \vec{u}\vec{u}^\top - 36I &= \myvec{-36 & 0 \\ 0 & 28}
\end{align}

Next, the linear coefficient:
\begin{align}
    -D^2\vec{u} + 2D^2\vec{A}
    &= -36\myvec{0\\8} + 72\myvec{0\\4} \\
    &= \myvec{0\\-288} + \myvec{0\\288} = \myvec{0\\0}
\end{align}

So there is no linear term.

Finally, the constant term:
\begin{align}
    \frac{D^4}{4} - 36\vec{A}^\top\vec{A}
    &= \frac{1296}{4} - 36(16) \\
    &= 324 - 576 \\
    &= -252
\end{align}

Therefore, the locus is given by
\begin{align}
    \vec{P}^\top\myvec{-36 & 0 \\ 0 & 28}\vec{P} - 252 &= 0
\end{align}

or equivalently
\begin{align}
    \vec{P}^\top\myvec{-36 & 0 \\ 0 & 28}\vec{P} &= 252
\end{align}

Expanding with $\vec{P}=\myvec{x\\y}$,
\begin{align}
    -36x^2 + 28y^2 &= 252
\end{align}

Dividing through,
\begin{align}
    \frac{y^2}{9} - \frac{x^2}{7} &= 1
\end{align}

Thus the locus is a hyperbola centered at the origin with equation
\begin{align}
    \frac{y^2}{9} - \frac{x^2}{7} = 1
\end{align}

\end{document}
