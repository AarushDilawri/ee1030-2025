\documentclass{beamer}
\usepackage[utf8]{inputenc}

\usetheme{Madrid}
\usecolortheme{default}
\usepackage{amsmath,amssymb,amsfonts,amsthm}
\usepackage{txfonts}
\usepackage{tkz-euclide}
\usepackage{listings}
\usepackage{adjustbox}
\usepackage{array}
\usepackage{tabularx}
\usepackage{gvv}
\usepackage{lmodern}
\usepackage{circuitikz}
\usepackage{tikz}
\usepackage{graphicx}
\usepackage{mathtools} 

\setbeamertemplate{page number in head/foot}[totalframenumber]

\title{5.2.4}
\date{\today}
\author{EE25BTECH11001 - Aarush Dilawri}

\begin{document}

\frame{\titlepage}

\begin{frame}{Problem}
\textbf{Question}:\\
Solve the system of equations
\begin{align}
6x-3y+10=0
\end{align}
\begin{align}
2x-y+9=0
\end{align}
\end{frame}

\begin{frame}{Solution Setup}
The equation of line:
\begin{align}
\vec{n}^\top\vec{x}=c
\end{align}

Line L:
\begin{align}
\myvec{6 & -3}\myvec{x \\ y}=-10
\end{align}

Line K:
\begin{align}
\myvec{2 & -1}\myvec{x \\ y}=-9
\end{align}
\end{frame}

\begin{frame}{Matrix Form}
These can be combined and written in matrix form:
\begin{align}
\myvec{6 & -3 \\ 2 & -1}\myvec{x \\ y} = \myvec{-10 \\ -9}
\end{align}

The following augmented matrix can be solved by Gaussian elimination:
\begin{align}
\augvec{2}{1}{6 & -3 & -10 \\ 2 & -1 & -9} \xleftrightarrow{R_2 \leftarrow 3R_2 - R_1} \augvec{2}{1}{6 & -3 & -10 \\ 0 & 0 & -17}
\end{align}
\end{frame}

\begin{frame}{Rank and Solution}
The rank of coefficient matrix is $1$ whereas the rank of the Augmented matrix is $0$. Thus, by Rouche–Capelli theorem:
\begin{align}
    \text{If rank}\brak{\vec{A}} \neq \text{rank}\brak{[\vec{A} \mid b]}\quad\text{, then the system has no solution.}
\end{align}

Hence, there are no solutions to this system of equations.
\end{frame}

\begin{frame}{Figure}
\begin{figure}[h!]
    \centering
    \includegraphics[height=0.5\textheight, keepaspectratio]{figs/fig.png}
    \label{figure_1}
\end{figure}
\end{frame}

\begin{frame}[fragile]{C Code (code.c)}
\begin{lstlisting}[language=C]
#include <stdio.h>
1, double b1, double c1,
                 double a2, double b2, double c2,
                 double *x, double *y) {
    double det = a1*b2 - a2*b1;
    if(det != 0) {
        *x = (b1*c2 - b2*c1) / det;
        *y = (c1*a2 - c2*a1) / det;
        return 1; // unique solution
    }
    // det = 0, check proportionality
    if(a1*b2 == a2*b1 && a1*c2 == a2*c1 && b1*c2 == b2*c1) {
        return 2; // infinite solutions
    }
    return 0; // no solution
}

\end{lstlisting}
\end{frame}

\begin{frame}[fragile]{Python Code (code.py)}
\begin{lstlisting}[language=Python]
# code.py
import matplotlib.pyplot as plt
import numpy as np

# Example equations: 6x - 3y + 10 = 0, 2x - y + 9 = 0
a1, b1, c1 = 6, -3, 10
a2, b2, c2 = 2, -1, 9

det = a1*b2 - a2*b1
xs = np.linspace(-10, 10, 400)

if det != 0:
    x = (b1*c2 - b2*c1) / det
    y = (c1*a2 - c2*a1) / det
    res = 1


\end{lstlisting}
\end{frame}

\begin{frame}[fragile]{Python Code (code.py)}
\begin{lstlisting}[language=Python]
elif (a1*b2 == a2*b1 and a1*c2 == a2*c1 and b1*c2 == b2*c1):
    res = 2
else:
    res = 0

# Plot lines
y1 = -(a1*xs + c1)/b1
y2 = -(a2*xs + c2)/b2
plt.plot(xs, y1, label=f"{a1}x+{b1}y+{c1}=0")
plt.plot(xs, y2, label=f"{a2}x+{b2}y+{c2}=0")

\end{lstlisting}
\end{frame}


\begin{frame}[fragile]{Python Code (code.py)}
\begin{lstlisting}[language=Python]

if res == 1:
    plt.scatter([x],[y], color="red", zorder=5, label=f"Unique ({x:.2f},{y:.2f})")
elif res == 2:
    plt.title("Infinite solutions (same line)")
else:
    plt.title("No solution (parallel lines)")

plt.legend()
plt.grid(True)
plt.show()
\end{lstlisting}
\end{frame}
\begin{frame}[fragile]{Python Code (nativecode.py)}
\begin{lstlisting}[language=Python]
# nativecode.py
import ctypes
import matplotlib.pyplot as plt
import numpy as np

# Load shared library
lib = ctypes.CDLL("./code.so")

lib.solve_linear.argtypes = [ctypes.c_double, ctypes.c_double, ctypes.c_double,
                             ctypes.c_double, ctypes.c_double, ctypes.c_double,
                             ctypes.POINTER(ctypes.c_double), ctypes.POINTER(ctypes.c_double)]
lib.solve_linear.restype = ctypes.c_int




\end{lstlisting}
\end{frame}
\begin{frame}[fragile]{Python Code (nativecode.py)}
\begin{lstlisting}[language=Python]
# Example equations: 6x - 3y + 10 = 0, 2x - y + 9 = 0
a1, b1, c1 = 6, -3, 10
a2, b2, c2 = 2, -1, 9
x = ctypes.c_double()
y = ctypes.c_double()

res = lib.solve_linear(a1,b1,c1,a2,b2,c2,ctypes.byref(x),ctypes.byref(y))

# Plot
xs = np.linspace(-10, 10, 400)
y1 = -(a1*xs + c1)/b1
y2 = -(a2*xs + c2)/b2

plt.plot(xs, y1, label=f"{a1}x+{b1}y+{c1}=0")
plt.plot(xs, y2, label=f"{a2}x+{b2}y+{c2}=0")



\end{lstlisting}
\end{frame}

\begin{frame}[fragile]{Python Code (nativecode.py)}
\begin{lstlisting}[language=Python]
if res == 1:
    plt.scatter([x.value],[y.value], color="red", zorder=5,
                label=f"Unique ({x.value:.2f},{y.value:.2f})")
elif res == 2:
    plt.title("Infinite solutions (same line)")
else:
    plt.title("No solution (parallel lines)")

plt.legend()
plt.grid(True)
plt.show()
\end{lstlisting}
\end{frame}

\end{document}
