\documentclass{beamer}
\usepackage[utf8]{inputenc}

\usetheme{Madrid}
\usecolortheme{default}
\usepackage{amsmath,amssymb,amsfonts,amsthm}
\usepackage{txfonts}
\usepackage{tkz-euclide}
\usepackage{listings}
\usepackage{adjustbox}
\usepackage{array}
\usepackage{tabularx}
\usepackage{gvv}
\usepackage{lmodern}
\usepackage{circuitikz}
\usepackage{tikz}
\usepackage{graphicx}
\usepackage{mathtools} 
\setbeamertemplate{page number in head/foot}[totalframenumber]

\title{9.4.13}
\date{\today}
\author{EE25BTECH11001 - Aarush Dilawri}

\begin{document}

\frame{\titlepage}
\begin{frame}{Question}
\textbf{Question:}\\
Find the roots of the following quadratic equation graphically
\begin{align}
    x^2 -2x = \brak{-2}\brak{3-x}
\end{align}
\end{frame}

\begin{frame}{Solution}
\textbf{Solution:}\\
\begin{align}
    y = x^2 -2x - \brak{-2}\brak{3-x}\\
    y = x^2 - 4x + 6 =0
\end{align}
This quadratic can be represented as a conic in matrix form:
\begin{align}
   \vec{x}^{\top}\vec{V}\vec{x} + 2\vec{u}^{\top}\vec{x} + f = 0\\ 
   \vec{V} = \myvec{1 & 0 \\ 0 & 0} , \vec{u} = \myvec{-2 \\0} ,
   f = 6
\end{align}
\end{frame}

\begin{frame}{Solution}
To find the roots, we find the points of intersection of the conic with the x-axis.
\begin{align}
\vec{x} = \vec{h} + k\vec{m}    
\end{align}
\begin{align}
\vec{h}=\myvec{0 \\ 0}, \vec{m} = \myvec{1 \\ 0}
\end{align}
\end{frame}

\begin{frame}{Solution}
The value of $k$ can be found out by solving the line and conic equation

\begin{align}
\brak{\vec{h} + k \vec{m}}^{\top} \vec{V} \brak{\vec{h} + k \vec{m}} + 2\vec{u}^{\top} \brak{\vec{h} + k \vec{m}} + f &= 0 \\
\implies k^{2} \vec{m}^{\top}\vec{V}\vec{m} + 2k \vec{m}^{\top} \brak{\vec{V}\vec{h} + \vec{u}} + \vec{h}^{\top}\vec{V}\vec{h} + 2\vec{u}^{\top}\vec{h} + f &= 0 \\
\text{or, } k^{2} \vec{m}^{\top}\vec{V}\vec{m} + 2k \vec{m}^{\top} \brak{\vec{V}\vec{h} + \vec{u}} + g\brak{\vec{h}} &= 0
\end{align}
\end{frame}

\begin{frame}{Solution}
    Solving the above quadratic gives the equation
\begin{align}
k = \frac{1}{\vec{m}^{\top}\vec{V}\vec{m}}
\brak{
    -\vec{m}^{\top} \brak{\vec{V}\vec{h} + \vec{u}}
    \pm
    \sqrt{ \sbrak{\vec{m}^{\top}\brak{\vec{V}\vec{h} + \vec{u}}}^2
    - g\brak{\vec{h}} \ \brak{\vec{m}^{\top}\vec{V}\vec{m}} }
    }
\end{align}

Substituting the values in the above equation gives
\begin{align}
\therefore k =2 \pm i\sqrt{2}
\end{align}
\begin{align}
 k_1 = 2 + i\sqrt{2}\\
 k_2 = 2 - i\sqrt{2}
\end{align}
\begin{align}
    \vec{x} = \vec{h} + k\vec{m}
   = \myvec{2 + i\sqrt{2} \\ 0},\quad
     \myvec{2 - i\sqrt{2} \\ 0}
\end{align}
$\therefore$ The given quadratic equation has imaginary roots.
\end{frame}
\begin{frame}{Graphical Representation}
See Figure,
\begin{figure}[h!]
    \centering
    \includegraphics[height=0.5\textheight, keepaspectratio]{figs/fig.png}
    \label{figure_1}
\end{figure}
\end{frame}
\begin{frame}[fragile]{Codes}
\url{https://github.com/AarushDilawri/ee1030-2025/tree/main/ee25btech11001/MATGEO/9.4.13/codes}
\end{frame}
\end{document}