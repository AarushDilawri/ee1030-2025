\documentclass{beamer}
\usepackage[utf8]{inputenc}

\usetheme{Madrid}
\usecolortheme{default}
\usepackage{amsmath,amssymb,amsfonts,amsthm}
\usepackage{txfonts}
\usepackage{tkz-euclide}
\usepackage{listings}
\usepackage{adjustbox}
\usepackage{array}
\usepackage{tabularx}
\usepackage{gvv}
\usepackage{lmodern}
\usepackage{circuitikz}
\usepackage{tikz}
\usepackage{graphicx}

\setbeamertemplate{page number in head/foot}[totalframenumber]

\title{8.4.4}
\date{\today}
\author{EE25BTECH11001 - Aarush Dilawri}

\begin{document}

\frame{\titlepage}

\begin{frame}{Question}
\textbf{Question}:\\
Equation of the ellipse whose axes are the coordinates and which passes through the point $\brak{-3,1}$ and has eccentricity $\sqrt{\frac{2}{5}}$ is
\end{frame}
\begin{frame}{Solution}
\textbf{Solution:} \\
The general equation of a conic can be written as:

\begin{align}
\vec{x}^T \vec{V} \vec{x} + 2 \vec{u}^T \vec{x} + f = 0
\end{align}

Since the ellipse is centered at origin, we have

\begin{align}
\vec{u} = \myvec{0 \\ 0}, \quad f = -1
\end{align}
\end{frame}
\begin{frame}{Solution}
Let the major axis be along the X-axis:

\begin{align}
\vec{n} = \myvec{1 \\ 0}, \quad e = \sqrt{\frac{2}{5}}
\end{align}

Then, using the formula:

\begin{align}
\vec{V} = \norm{\vec{n}}^2 \vec{I} - e^2 \vec{n} \vec{n}^T
\end{align}

we get

\begin{align}
\vec{V} = \myvec{1 - \frac{2}{5} & 0 \\ 0 & 1} = \myvec{\frac{3}{5} & 0 \\ 0 & 1}
\end{align}
\end{frame}
    
\begin{frame}{Solution}
The ellipse passes through the point $\brak{-3,1}$, so scale $\vec{V}$ such that:

\begin{align}
\vec{x}_0^T \vec{V} \vec{x}_0 = 1, \quad \vec{x}_0 = \myvec{-3 \\ 1}
\end{align}

This gives

\begin{align}
\vec{V} = \frac{5}{32} \myvec{\frac{3}{5} & 0 \\ 0 & 1} = \myvec{\frac{3}{32} & 0 \\ 0 & \frac{5}{32}}
\end{align}
\end{frame}


\begin{frame}{Solution}
Hence, the equation of the ellipse is:

\begin{align}
\vec{x}^T \myvec{\frac{3}{32} & 0 \\ 0 & \frac{5}{32}} \vec{x} = 1
\end{align}

Or equivalently:

\begin{align}
3 x^2 + 5 y^2 = 32
\end{align}
\end{frame}

\begin{frame}{Figure}
\begin{center}
\includegraphics[width=0.6\columnwidth]{figs/fig.png}
\end{center}
\end{frame}
\begin{frame}[fragile]{C Code (code.c)}
\begin{lstlisting}[language=C]
#include <math.h>
int ellipse_equation(double *A, double *B, double *C,
                     double x0, double y0, double e)
{
    double V11 = 1.0 - e*e;
    double V22 = 1.0;

    double val = V11*x0*x0 + V22*y0*y0;
    double scale = 1.0 / val;

    *A = V11 * scale;
    *B = V22 * scale;
    *C = 1.0;

    return 0;
}
\end{lstlisting}
\end{frame}
\begin{frame}[fragile]{Python Code (code.py)}
\begin{lstlisting}[language=Python]
import numpy as np
import matplotlib.pyplot as plt

# Given parameters
x0, y0 = -3.0, 1.0
e = np.sqrt(2.0/5.0)

V11 = 1 - e**2
V22 = 1

val = V11*x0**2 + V22*y0**2
scale = 1.0 / val
A = V11 * scale
B = V22 * scale
C = 1.0

print(f"Ellipse equation (Python): {A:.6f} x^2 + {B:.6f} y^2 = {C:.6f}")
\end{lstlisting}
\end{frame}
\begin{frame}[fragile]{Python Code (code.py)}
\begin{lstlisting}[language=Python]
theta = np.linspace(0, 2*np.pi, 400)
a = np.sqrt(C/A)
b = np.sqrt(C/B)
x = a * np.cos(theta)
y = b * np.sin(theta)

plt.figure(figsize=(6,6))
plt.plot(x, y, label=f"{A:.2f} x² + {B:.2f} y² = {C:.0f}")
plt.scatter(x0, y0, color='red', label='(-3,1)')
plt.axhline(0, color='gray', linewidth=0.5)
plt.axvline(0, color='gray', linewidth=0.5)
plt.gca().set_aspect('equal', adjustable='box')
plt.legend()
plt.grid(True)
plt.title('Ellipse (Python only)')
plt.show()
\end{lstlisting}
\end{frame}
\begin{frame}[fragile]{Python Code (nativecode.py)}
\begin{lstlisting}[language=Python]
import ctypes
import numpy as np
import matplotlib.pyplot as plt

# Load the shared library
lib = ctypes.CDLL('./code.so')

# Define argument and return types
lib.ellipse_equation.argtypes = [ctypes.POINTER(ctypes.c_double),
                                 ctypes.POINTER(ctypes.c_double),
                                 ctypes.POINTER(ctypes.c_double),
                                 ctypes.c_double,
                                 ctypes.c_double,
                                 ctypes.c_double]
lib.ellipse_equation.restype = ctypes.c_int  # returning int now

\end{lstlisting}
\end{frame}
\begin{frame}[fragile]{Python Code (nativecode.py)}
\begin{lstlisting}[language=Python]
A = ctypes.c_double()
B = ctypes.c_double()
C = ctypes.c_double()
x0 = -3.0
y0 = 1.0
e = np.sqrt(2.0/5.0)

ret = lib.ellipse_equation(ctypes.byref(A), ctypes.byref(B), ctypes.byref(C), x0, y0, e)

if ret != 0:
    raise RuntimeError("Error computing ellipse coefficients in C.")

print(f"Ellipse equation from C: {A.value:.6f} x^2 + {B.value:.6f} y^2 = {C.value:.6f}")

theta = np.linspace(0, 2*np.pi, 400)
\end{lstlisting}
\end{frame}
\begin{frame}[fragile]{Python Code (nativecode.py)}
\begin{lstlisting}[language=Python]
a = np.sqrt(C.value/A.value)
b = np.sqrt(C.value/B.value)
x = a * np.cos(theta)
y = b * np.sin(theta)

plt.figure(figsize=(6,6))
plt.plot(x, y, label=f"{A.value:.2f} x² + {B.value:.2f} y² = {C.value:.0f}")
plt.scatter(x0, y0, color='red', label='(-3,1)')
plt.axhline(0, color='gray', linewidth=0.5)
plt.axvline(0, color='gray', linewidth=0.5)
plt.gca().set_aspect('equal', adjustable='box')
plt.legend()
plt.grid(True)
plt.title('Ellipse from C Library')
plt.show()
\end{lstlisting}
\end{frame}
\end{document}

