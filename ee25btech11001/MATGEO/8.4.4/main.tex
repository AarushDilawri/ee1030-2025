\let\negmedspace\undefined
\let\negthickspace\undefined
\documentclass[journal]{IEEEtran}
\usepackage[a5paper, margin=10mm, onecolumn]{geometry}
%\usepackage{lmodern} % Ensure lmodern is loaded for pdflatex
\usepackage{tfrupee} % Include tfrupee package

\setlength{\headheight}{1cm} % Set the height of the header box
\setlength{\headsep}{0mm}     % Set the distance between the header box and the top of the text

\usepackage{gvv-book}
\usepackage{gvv}
\usepackage{cite}
\usepackage{amsmath,amssymb,amsfonts,amsthm}
\usepackage{algorithmic}
\usepackage{graphicx}
\usepackage{textcomp}
\usepackage{xcolor}
\usepackage{txfonts}
\usepackage{listings}
\usepackage{enumitem}
\usepackage{mathtools}
\usepackage{gensymb}
\usepackage{comment}
\usepackage[breaklinks=true]{hyperref}
\usepackage{tkz-euclide} 
\usepackage{listings}
% \usepackage{gvv}                                        
\def\inputGnumericTable{}                                 
\usepackage[latin1]{inputenc}                                
\usepackage{color}                                            
\usepackage{array}                                            
\usepackage{longtable}                                       
\usepackage{calc}                                             
\usepackage{multirow}                                         
\usepackage{hhline}                                           
\usepackage{ifthen}                                           
\usepackage{lscape}
\begin{document}

\bibliographystyle{IEEEtran}
\vspace{3cm}

\title{8.4.4}
\author{EE25BTECH11001 - Aarush Dilawri}
% \maketitle
% \newpage
% \bigskip
{\let\newpage\relax\maketitle}

\renewcommand{\thefigure}{\theenumi}
\renewcommand{\thetable}{\theenumi}
\setlength{\intextsep}{10pt} % Space between text and floats
\textbf{Question}:\\
Equation of the ellipse whose axes are the coordinates and which passes through the point $\brak{-3,1}$ and has eccentricity $\sqrt{\frac{2}{5}}$ is\\

\textbf{Solution:} \\
The general equation of a conic can be written as:

\begin{align}
\vec{x}^T \vec{V} \vec{x} + 2 \vec{u}^T \vec{x} + f = 0
\end{align}

Since the ellipse is centered at origin, we have

\begin{align}
\vec{u} = \myvec{0 \\ 0}, \quad f = -1
\end{align}

Let the major axis be along the X-axis:

\begin{align}
\vec{n} = \myvec{1 \\ 0}, \quad e = \sqrt{\frac{2}{5}}
\end{align}

Then, using the formula:

\begin{align}
\vec{V} = \norm{\vec{n}}^2 \vec{I} - e^2 \vec{n} \vec{n}^T
\end{align}

we get

\begin{align}
\vec{V} = \myvec{1 - \frac{2}{5} & 0 \\ 0 & 1} = \myvec{\frac{3}{5} & 0 \\ 0 & 1}
\end{align}

The ellipse passes through the point $\brak{-3,1}$, so scale $\vec{V}$ such that:

\begin{align}
\vec{x}_0^T \vec{V} \vec{x}_0 = 1, \quad \vec{x}_0 = \myvec{-3 \\ 1}
\end{align}

This gives

\begin{align}
\vec{V} = \frac{5}{32} \myvec{\frac{3}{5} & 0 \\ 0 & 1} = \myvec{\frac{3}{32} & 0 \\ 0 & \frac{5}{32}}
\end{align}

Hence, the equation of the ellipse is:

\begin{align}
\vec{x}^T \myvec{\frac{3}{32} & 0 \\ 0 & \frac{5}{32}} \vec{x} = 1
\end{align}

Or equivalently:

\begin{align}
3 x^2 + 5 y^2 = 32
\end{align}


See Fig. 0 ,
\begin{figure}[H]
\begin{center}
\includegraphics[width=0.6\columnwidth]{figs/fig.png}
\end{center}
\caption{}
\label{fig:Fig1}
\end{figure}
\end{document}
