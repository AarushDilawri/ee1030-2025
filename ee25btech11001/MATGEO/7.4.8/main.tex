\let\negmedspace\undefined
\let\negthickspace\undefined
\documentclass[journal]{IEEEtran}
\usepackage[a5paper, margin=10mm, onecolumn]{geometry}
%\usepackage{lmodern} % Ensure lmodern is loaded for pdflatex
\usepackage{tfrupee} % Include tfrupee package

\setlength{\headheight}{1cm} % Set the height of the header box
\setlength{\headsep}{0mm}     % Set the distance between the header box and the top of the text

\usepackage{gvv-book}
\usepackage{gvv}
\usepackage{cite}
\usepackage{amsmath,amssymb,amsfonts,amsthm}
\usepackage{algorithmic}
\usepackage{graphicx}
\usepackage{textcomp}
\usepackage{xcolor}
\usepackage{txfonts}
\usepackage{listings}
\usepackage{enumitem}
\usepackage{mathtools}
\usepackage{gensymb}
\usepackage{comment}
\usepackage[breaklinks=true]{hyperref}
\usepackage{tkz-euclide} 
\usepackage{listings}
% \usepackage{gvv}                                        
\def\inputGnumericTable{}                                 
\usepackage[latin1]{inputenc}                                
\usepackage{color}                                            
\usepackage{array}                                            
\usepackage{longtable}                                       
\usepackage{calc}                                             
\usepackage{multirow}                                         
\usepackage{hhline}                                           
\usepackage{ifthen}                                           
\usepackage{lscape}
\usepackage{circuitikz}
\tikzstyle{block} = [rectangle, draw, fill=blue!20, 
    text width=4em, text centered, rounded corners, minimum height=3em]
\tikzstyle{sum} = [draw, fill=blue!10, circle, minimum size=1cm, node distance=1.5cm]
\tikzstyle{input} = [coordinate]
\tikzstyle{output} = [coordinate]


\begin{document}

\bibliographystyle{IEEEtran}
\vspace{3cm}

\title{7.4.8}
\author{EE25BTECH11001 - Aarush Dilawri}
\maketitle
% \newpage
% \bigskip
{\let\newpage\relax\maketitle}

\renewcommand{\thefigure}{\theenumi}
\renewcommand{\thetable}{\theenumi}
\setlength{\intextsep}{10pt} % Space between text and floats


\numberwithin{equation}{enumi}
\numberwithin{figure}{enumi}
\renewcommand{\thetable}{\theenumi}

\textbf{Question}:\\
For each natural number $k$, let $C_k$ denote the circle with radius $k$ centimetres and
centre at the origin. On the circle $C_k$ , a particle moves $k$ centimetres in the counter-
clockwise direction. After completing its motion on $C_k$ , the particle moves to $C_{k+1}$ in the radial direction. The motion of the particle continues in this manner. The particle
starts at $\brak{1,0}$. If the particle crosses the positive direction of the $X$ axis for the first time on the Circle $C_n$, then $n$ = \_

\solution \\

\begin{align}
    \text{Let }\; \vec{p}_0 &= \myvec{1 \\ 0} 
\end{align}

We model a rotation by an angle $\theta$ using the rotation matrix
\begin{align}
    \vec{R}\brak{\theta} &= \myvec{\cos\theta & -\sin\theta \\[6pt] \sin\theta & \cos\theta}
\end{align}

Note the group property of rotations:
\begin{align}
    \vec{R}\brak{\theta_1}\,\vec{R}\brak{\theta_2} &= \vec{R}\brak{\theta_1+\theta_2},
    \qquad \vec{R}\brak{\theta}^k = \vec{R}\brak{k\theta}.
\end{align}

On the circle $C_k$ the particle moves an arc of length $k$ on a circle of radius $k$, 
so the angular increment on $C_k$ is
\begin{align}
    \Delta\theta_k &= \frac{\text{arc length}}{\text{radius}} \;=\; \frac{k}{k} \;=\; 1 \quad\text{(radian).}
\end{align}

Thus each circular motion rotates the particle by $1$ radian.  We track the position of the particle at the instant it finishes its motion on $C_k$ (that is, after the arc motion but before the radial jump to $C_{k+1}$).  
Starting at $\vec{p}_0$ on $C_1$, after finishing $C_1$ the position is
\begin{align}
    \vec{P}_1 &= 1\;\vec{R}\brak{1}\,\vec{p}_0.
\end{align}
Then the particle moves radially to $C_2$, scaling the radius from $1$ to $2$, so just before moving on $C_2$ the vector is $2\vec{R}(1)\vec{p}_0$. After moving on $C_2$ (an additional rotation by $1$) the particle is at
\begin{align}
    \vec{P}_2 &= 2\;\vec{R}(1)\vec{R}(1)\,\vec{p}_0 \;=\; 2\;\vec{R}(2)\,\vec{p}_0.
\end{align}
By induction, after finishing its motion on $C_k$ the particle is at
\begin{align}
    \vec{P}_k &= k\;\vec{R}(k)\,\vec{p}_0.
\end{align}

Therefore the angular coordinate of the particle after completing $C_k$ is exactly $k$ radians.
The motion on $C_n$ runs the angle from $(n-1)$ to $n$ (radians).  
Hence the particle crosses the positive $x$-axis during the motion on $C_n$ precisely when some integer multiple of $2\pi$ lies in the interval $(n-1,n]$, i.e. when there exists $m\in\mathbb{N}$ such that
\begin{align}
    n-1 \;<\; 2\pi m \;\le\; n .
\end{align}

We look for the smallest natural number $n$ for which this happens.  
Take $m=1$ (the first positive multiple of $2\pi$). Compute
\begin{align}
    2\pi &\approx 6.283185307\ldots
\end{align}
and observe
\begin{align}
    6 \;<\; 2\pi \;\le\; 7.
\end{align}
Thus $2\pi$ lies in the interval $(6,7]$, so the condition holds for $n=7$ (with $m=1$).  
For any $n\le 6$ the interval $(n-1,n]$ is contained in $[0,6]$ and cannot contain $2\pi\approx 6.283\ldots$.

Therefore the particle crosses the positive $x$-axis for the first time while moving on $C_n$ with
\begin{align}
    \boxed{n \;=\; 7.}
\end{align}

See Figure,
\begin{figure}[h!]
    \centering
    \includegraphics[height=0.5\textheight, keepaspectratio]{figs/fig.png}
    \label{figure_1}
\end{figure}
\end{document}
