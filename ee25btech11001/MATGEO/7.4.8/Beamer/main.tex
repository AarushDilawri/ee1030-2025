\documentclass{beamer}
\usepackage[utf8]{inputenc}

\usetheme{Madrid}
\usecolortheme{default}
\usepackage{amsmath,amssymb,amsfonts,amsthm}
\usepackage{txfonts}
\usepackage{tkz-euclide}
\usepackage{listings}
\usepackage{adjustbox}
\usepackage{array}
\usepackage{tabularx}
\usepackage{gvv}
\usepackage{lmodern}
\usepackage{circuitikz}
\usepackage{tikz}
\usepackage{graphicx}

\setbeamertemplate{page number in head/foot}[totalframenumber]

\title{7.4.8}
\date{\today}
\author{EE25BTECH11001 - Aarush Dilawri}

\begin{document}

\frame{\titlepage}

\begin{frame}{Question}
\textbf{Question}:\\
For each natural number $k$, let $C_k$ denote the circle with radius $k$ centimetres and
centre at the origin. On the circle $C_k$ , a particle moves $k$ centimetres in the counter-
clockwise direction. After completing its motion on $C_k$ , the particle moves to $C_{k+1}$ in the radial direction. The motion of the particle continues in this manner. The particle
starts at $\brak{1,0}$. If the particle crosses the positive direction of the $X$ axis for the first time on the Circle $C_n$, then $n$ = \_
\end{frame}

\begin{frame}{Solution}
\textbf{Solution:} \\
\begin{align}
    \text{Let }\; \vec{p}_0 &= \myvec{1 \\ 0} 
\end{align}

We model a rotation by an angle $\theta$ using the rotation matrix
\begin{align}
    \vec{R}\brak{\theta} &= \myvec{\cos\theta & -\sin\theta \\[6pt] \sin\theta & \cos\theta}
\end{align}

Note the group property of rotations:
\begin{align}
    \vec{R}\brak{\theta_1}\,\vec{R}\brak{\theta_2} &= \vec{R}\brak{\theta_1+\theta_2},
    \qquad \vec{R}\brak{\theta}^k = \vec{R}\brak{k\theta}.
\end{align}
\end{frame}

\begin{frame}{Solution}
On the circle $C_k$ the particle moves an arc of length $k$ on a circle of radius $k$, 
so the angular increment on $C_k$ is
\begin{align}
    \Delta\theta_k &= \frac{\text{arc length}}{\text{radius}} \;=\; \frac{k}{k} \;=\; 1 \quad\text{(radian).}
\end{align}

Thus each circular motion rotates the particle by $1$ radian.  We track the position of the particle at the instant it finishes its motion on $C_k$ (that is, after the arc motion but before the radial jump to $C_{k+1}$).  
\end{frame}

\begin{frame}{Solution}
Starting at $\vec{p}_0$ on $C_1$, after finishing $C_1$ the position is
\begin{align}
    \vec{P}_1 &= 1\;\vec{R}\brak{1}\,\vec{p}_0.
\end{align}
Then the particle moves radially to $C_2$, scaling the radius from $1$ to $2$, so just before moving on $C_2$ the vector is $2\vec{R}\brak{1}\vec{p}_0$. After moving on $C_2$ (an additional rotation by $1$) the particle is at
\begin{align}
    \vec{P}_2 &= 2\;\vec{R}\brak{1}\vec{R}\brak{1}\,\vec{p}_0 \;=\; 2\;\vec{R}\brak{2}\,\vec{p}_0.
\end{align}
By induction, after finishing its motion on $C_k$ the particle is at
\begin{align}
    \vec{P}_k &= k\;\vec{R}\brak{k}\,\vec{p}_0.
\end{align}
\end{frame}

\begin{frame}{Solution}
Therefore the angular coordinate of the particle after completing $C_k$ is exactly $k$ radians.
The motion on $C_n$ runs the angle from $(n-1)$ to $n$ (radians).  
Hence the particle crosses the positive $x$-axis during the motion on $C_n$ precisely when some integer multiple of $2\pi$ lies in the interval $(n-1,n]$, i.e. when there exists $m\in\mathbb{N}$ such that
\begin{align}
    n-1 \;<\; 2\pi m \;\le\; n .
\end{align}
\end{frame}

\begin{frame}{Solution}
We look for the smallest natural number $n$ for which this happens.  
Take $m=1$ (the first positive multiple of $2\pi$). Compute
\begin{align}
    2\pi &\approx 6.283185307\ldots
\end{align}
and observe
\begin{align}
    6 \;<\; 2\pi \;\le\; 7.
\end{align}
Thus $2\pi$ lies in the interval $(6,7]$, so the condition holds for $n=7$ (with $m=1$).  
For any $n\le 6$ the interval $(n-1,n]$ is contained in $[0,6]$ and cannot contain $2\pi\approx 6.283\ldots$.

Therefore the particle crosses the positive $x$-axis for the first time while moving on $C_n$ with
\begin{align}
    \boxed{n \;=\; 7.}
\end{align}
\end{frame}

\begin{frame}{Graphical Representation}
\begin{figure}[h!]
    \centering
    \includegraphics[height=0.5\textheight, keepaspectratio]{figs/fig.png}
    \label{figure_1}
\end{figure}
\end{frame}

% --- Code frames below remain unchanged ---
% (no R() occurrences there)

\begin{frame}[fragile]{C Code (code.c)}
\begin{lstlisting}[language=C]
#include <stdio.h>
#include <math.h>

void particle_endpoints(int n, double *px, double *py, double *theta_out) {
    double theta = 0.0;
    for (int k = 1; k <= n; ++k) {
        double r = (double)k;
        // Arc length on C_k = k ⇒ Δθ = 1 rad
        double delta = 1.0;
        theta += delta;
        px[k-1] = r * cos(theta);
        py[k-1] = r * sin(theta);
        if (theta_out != NULL) theta_out[k-1] = theta;
    }
}
\end{lstlisting}
\end{frame}

% (The rest of the Python and native code frames remain identical)

\end{document}


