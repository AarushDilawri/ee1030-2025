\documentclass{beamer}
\usepackage[utf8]{inputenc}

\usetheme{Madrid}
\usecolortheme{default}
\usepackage{amsmath,amssymb,amsfonts,amsthm}
\usepackage{txfonts}
\usepackage{tkz-euclide}
\usepackage{listings}
\usepackage{adjustbox}
\usepackage{array}
\usepackage{tabularx}
\usepackage{gvv}
\usepackage{lmodern}
\usepackage{circuitikz}
\usepackage{tikz}
\usepackage{graphicx}

\setbeamertemplate{page number in head/foot}[totalframenumber]

\title{4.9.3}
\date{\today}
\author{EE25BTECH11001 - Aarush Dilawri}

\begin{document}

\frame{\titlepage}


\begin{frame}{Question}
\textbf{Question}:\\
Find the equations of the two lines passing through the origin which intersect the line 
\[
\frac{x-3}{2} = \frac{y-3}{1} = \frac{z}{1}
\]
at angles of $\tfrac{\pi}{3}$ each.
\end{frame}

\begin{frame}{Solution}
The given line can be expressed as
\begin{align}
    \vec{x} &= \vec{h} + \kappa\vec{m}\\
    \text{where}\quad \vec{h} &= \myvec{3\\3\\0}, \quad \vec{m} = \myvec{2\\1\\1}
\end{align}
Any point $\vec{P}$ on this line can be given as 
\begin{align}
    \vec{P} = \vec{h} + \kappa\vec{m}
\end{align}
The line through the origin and $\vec{P}$ will have direction vector $\vec{P}$.
\end{frame}

\begin{frame}{Solution}
Since the angle between $\vec{m}$ and $\vec{P}$ is $\tfrac{\pi}{3}$,
\begin{align}
\cos \theta &= \frac{\vec{m}^\top \vec{P}}{\|\vec{m}\| \,\|\vec{P}\|} \\
\implies \left( \vec{m}^\top \vec{P} \right)^2 &= \cos^2\theta\,(\vec{m}^\top\vec{m})(\vec{P}^\top\vec{P}).
\end{align}
Substituting $\vec{P} = \vec{h} + \kappa\vec{m}$ and solving, we get
\begin{align}
    27\kappa^2 + 81\kappa + 54 = 0\\
    \kappa^2 + 3\kappa + 2 =0\\
    \implies \kappa = -1,-2
\end{align}
\end{frame}

\begin{frame}{Solution}
Therefore, the direction vectors of the lines are
\begin{align}
    \myvec{1\\2\\-1} \quad \text{and} \quad \myvec{-1\\1\\-2}
\end{align}
Thus, the equations of the lines are
\begin{align}
    \vec{x} = \lambda\myvec{1\\2\\-1} \quad \text{and} \quad 
    \vec{x} = \mu\myvec{-1\\1\\-2}
\end{align}
\end{frame}
\begin{frame}{Figure}
\begin{figure}[h!]
    \centering
    \includegraphics[height=0.5\textheight, keepaspectratio]{figs/fig.png}
\end{figure}
\end{frame}


\end{document}