\let\negmedspace\undefined
\let\negthickspace\undefined
\documentclass[journal]{IEEEtran}
\usepackage[a5paper, margin=10mm, onecolumn]{geometry}
%\usepackage{lmodern} % Ensure lmodern is loaded for pdflatex
\usepackage{tfrupee} % Include tfrupee package

\setlength{\headheight}{1cm} % Set the height of the header box
\setlength{\headsep}{0mm}     % Set the distance between the header box and the top of the text

\usepackage{gvv-book}
\usepackage{gvv}
\usepackage{cite}
\usepackage{amsmath,amssymb,amsfonts,amsthm}
\usepackage{algorithmic}
\usepackage{graphicx}
\usepackage{textcomp}
\usepackage{xcolor}
\usepackage{txfonts}
\usepackage{listings}
\usepackage{enumitem}
\usepackage{mathtools}
\usepackage{gensymb}
\usepackage{comment}
\usepackage[breaklinks=true]{hyperref}
\usepackage{tkz-euclide} 
\usepackage{listings}
% \usepackage{gvv}                                        
\def\inputGnumericTable{}                                 
\usepackage[latin1]{inputenc}                                
\usepackage{color}                                            
\usepackage{array}                                            
\usepackage{longtable}                                       
\usepackage{calc}                                             
\usepackage{multirow}                                         
\usepackage{hhline}                                           
\usepackage{ifthen}                                           
\usepackage{lscape}
\usepackage{circuitikz}
\tikzstyle{block} = [rectangle, draw, fill=blue!20, 
    text width=4em, text centered, rounded corners, minimum height=3em]
\tikzstyle{sum} = [draw, fill=blue!10, circle, minimum size=1cm, node distance=1.5cm]
\tikzstyle{input} = [coordinate]
\tikzstyle{output} = [coordinate]


\begin{document}

\bibliographystyle{IEEEtran}
\vspace{3cm}

\title{4.9.3}
\author{EE25BTECH11001 - Aarush Dilawri}
\maketitle
% \newpage
% \bigskip
{\let\newpage\relax\maketitle}

\renewcommand{\thefigure}{\theenumi}
\renewcommand{\thetable}{\theenumi}
\setlength{\intextsep}{10pt} % Space between text and floats


\numberwithin{equation}{enumi}
\numberwithin{figure}{enumi}
\renewcommand{\thetable}{\theenumi}

\textbf{Question}:\\
Find the equations of the two lines passing through the origin which intersect the line $\frac{x-3}{2} = \frac{y-3}{1} = \frac{z}{1}$ at angles of $\frac{\pi}{3}$ each.

\solution \\
The given line can be expressed as
\begin{align}
    \vec{x} = \vec{h} + \kappa\vec{m}\\
    \text{where}\quad \vec{h} = \myvec{3\\3\\0}\quad \text{and} \quad\vec{m} = \myvec{2\\1\\1}
\end{align}
Any point $\vec{P}$ on this line can be given as 
\begin{align}
    \vec{P} = \vec{h} + \kappa\vec{m}
\end{align}
The line through the origin and $\vec{P}$ will have direction vector $\vec{P}$.\\\\
Since the angle between $\vec{m}$ and $\vec{P}$ is $\tfrac{\pi}{3}$,
\begin{align}
\cos \theta &= \frac{\vec{m}^\top \vec{P}}{\|\vec{m}\| \,\|\vec{P}\|} \\
\implies \left( \vec{m}^\top \vec{P} \right)^2 &= \cos^2\theta\,(\vec{m}^\top\vec{m})(\vec{P}^\top\vec{P}).
\end{align}
Substituting $\vec{P} = \vec{h} + \kappa\vec{m}$ and solving, we get a quadratic equation in $\kappa$
\begin{align}
    \kappa^2\brak{\vec{m}^\top\vec{m}}^2\sin^2\theta + 2\kappa\brak{\vec{m}^\top\vec{m}}\brak{\vec{m}^\top\vec{h}}\sin^2\theta + \brak{\vec{m}^\top\vec{h}}^2 - \vec{m}^\top\vec{m}\cos^2\theta\vec{h}^\top\vec{h} = 0
\end{align}
Plugging in the values,
\begin{align}
    27\kappa^2 + 81\kappa + 54 = 0\\
    \kappa^2 + 3\kappa + 2 =0\\
    \implies \kappa = -1,-2
\end{align}
Therefore, the direction vectors of the lines are
\begin{align}
    \vec{x} = \myvec{3\\3\\0}-\myvec{2\\1\\1}=\myvec{1\\2\\-1} \quad \text{and} \quad \vec{x} = \myvec{3\\3\\0}-2\myvec{2\\1\\1}=\myvec{-1\\1\\-2}
\end{align}
Therefore, the equations of the lines are
\begin{align}
    \vec{x} = \lambda\myvec{1\\2\\-1} \quad \text{and} \quad \vec{x} = \mu\myvec{-1\\1\\-2}
\end{align}
See Figure,
\begin{figure}[h!]
    \centering
    \includegraphics[height=0.5\textheight, keepaspectratio]{figs/fig.png}
    %\caption{Direction and Normal Vectors}
\end{figure}
\end{document}
