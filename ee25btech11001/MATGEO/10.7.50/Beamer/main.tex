\documentclass{beamer}
\usepackage[utf8]{inputenc}

\usetheme{Madrid}
\usecolortheme{default}
\usepackage{amsmath,amssymb,amsfonts,amsthm}
\usepackage{txfonts}
\usepackage{tkz-euclide}
\usepackage{listings}
\usepackage{adjustbox}
\usepackage{array}
\usepackage{tabularx}
\usepackage{gvv}
\usepackage{lmodern}
\usepackage{circuitikz}
\usepackage{tikz}
\usepackage{graphicx}

\setbeamertemplate{page number in head/foot}[totalframenumber]

\title{10.7.50}
\date{\today}
\author{EE25BTECH11001 - Aarush Dilawri}

\begin{document}

\frame{\titlepage}

\begin{frame}{Question}
\textbf{Question}:\\[6pt]
Consider the family of circles $x^2 + y^2 = r^2 , 2 < r < 5$. If in the first quadrant, the
common tangent to a circle of this family and the ellipse $4x^2 + 25y^2 = 100$ meets
the coordinate axes at $\vec{A}$ and $\vec{B}$, then find the equation of the locus of the midpoint of $AB$.
\end{frame}

% ------------------ Solution ------------------
\begin{frame}{Solution}
\textbf{Solution:}\\
\begin{align}
\text{The family of circles is } 
\vec{X}^\top\vec{X} = r^2,\qquad 2<r<5,
\end{align}
\begin{align}
\text{and the ellipse is } 
\vec{X}^\top\vec{V}\vec{X} = 100,\quad
\vec{V} = \myvec{4 & 0 \\ 0 & 25}.
\end{align}

Let the common tangent meet the coordinate axes at 
\begin{align}
\vec{A} = a\vec{e}_1,\qquad \vec{B} = b\vec{e}_2,
\qquad a,b>0.
\end{align}

The equation of the line passing through $\vec{A} \text{ and } \vec{B}$ can be written as 
\begin{align}
\frac{\vec{e}_1^\top\vec{X}}{a}+\frac{\vec{e}_2^\top\vec{X}}{b}=1.
\end{align}
\end{frame}

% ------------------ Next Slide ------------------
\begin{frame}{Solution}
This is of the form  $\vec{n}^\top\vec{X}=c$, with 
\begin{align}
\vec{n} = \myvec{\dfrac{1}{a}\\[6pt]\dfrac{1}{b}},
\quad c=1.
\end{align}

Let the midpoint of $\vec{A} \text{ and } \vec{B}$ be 
\begin{align}
\vec{m} = \frac{\vec{A}+\vec{B}}{2}.
\end{align}

From this,
\begin{align}
a = 2\,\vec{e}_1^\top\vec{m},\quad
b = 2\,\vec{e}_2^\top\vec{m}.
\end{align}
\end{frame}

\begin{frame}{Solution}
The ellipse is
\begin{align}
\vec{x}^\top \vec{V}\vec{x} = 100.
\end{align}

The line is
\begin{align}
\vec{n}^\top \vec{x} = c.
\end{align}

Suppose $\vec{x}_0 = \alpha \vec{n}$ is a solution.  
Then
\begin{align}
\vec{n}^\top \vec{x}_0 = \alpha \vec{n}^\top \vec{n} = c
\implies \alpha = \frac{c}{\vec{n}^\top \vec{n}}.
\end{align}
\end{frame}

\begin{frame}{Solution}
So a particular solution is
\begin{align}
\vec{x}_0 = \frac{c}{\vec{n}^\top \vec{n}} \vec{n}.
\end{align}

Any point on the line can be written as
\begin{align}
\vec{x} = \vec{x}_0 + \mu \vec{m},
\end{align}
where $\vec{m}$ is a direction vector satisfying
\begin{align}
\vec{n}^\top \vec{m} = 0.
\end{align}

Substitute into $\vec{x}^\top \vec{V}\vec{x} = 100$:
\begin{align}
\brak{\vec{x}_0 + \mu \vec{m}}^\top \vec{V} \brak{\vec{x}_0 + \mu \vec{m}} = 100.
\end{align}
\end{frame}

\begin{frame}{Solution}
Expanding,
\begin{align}
\vec{x}_0^\top \vec{V}\vec{x}_0
+ 2\mu \vec{m}^\top \vec{V}\vec{x}_0
+ \mu^2 \vec{m}^\top \vec{V}\vec{m} = 100.
\end{align}

This is a quadratic in $\mu$

For tangency, discriminant $=0$:
That is,
\begin{align}
\brak{2\vec{m}^\top \vec{V}\vec{x}_0}^2
- 4\brak{\vec{m}^\top \vec{V}\vec{m}}
\brak{\vec{x}_0^\top \vec{V}\vec{x}_0 - 100} = 0.
\end{align}
\end{frame}
\begin{frame}{Solution}
After simplification using
\begin{align}
\vec{x}_0 = \frac{c}{\vec{n}^\top \vec{n}} \vec{n}, \qquad 
\vec{n}^\top \vec{m} = 0,
\end{align}
the condition reduces to
\begin{align}
c^2 = 100 \,\vec{n}^\top \vec{V}^{-1}\vec{n}.
\end{align}
\end{frame}
% ------------------ Next Slide ------------------
\begin{frame}{Solution}
\begin{align*}
\therefore\text{For the line } \vec{n}^\top\vec{X}=c \text{ to be tangent to } \\
\vec{X}^\top\vec{V}\vec{X}=100,\text{ the condition is } 
c^2 = 100\,\vec{n}^\top\vec{V}^{-1}\vec{n}.
\end{align*}

\begin{align}
\text{Here } c=1,\quad
\vec{V}^{-1}=\myvec{\tfrac{1}{4} & 0 \\ 0 & \tfrac{1}{25}}.
\text{ Substituting gives }
1 = 100\Big(\frac{1}{4a^2}+\frac{1}{25b^2}\Big).
\end{align}

\begin{align}
\Rightarrow \frac{25}{a^2}+\frac{4}{b^2}=1.
\end{align}
\end{frame}

% ------------------ Next Slide ------------------
\begin{frame}{Solution}
Also, for the line $\vec{n}^\top\vec{X}=c$ to be tangent to the circle $\vec{X}^\top\vec{X}=r^2$,\\ 
the distance from the origin must equal $r$.
\begin{align}
    \frac{|c|}{\|\vec{n}\|}=r.\\
    \text{With } c=1 \text{ this gives } \|\vec{n}\|^2 = \frac{1}{r^2}.
    \text{ So }
    \frac{1}{a^2}+\frac{1}{b^2}=\frac{1}{r^2}.
\end{align}

We now express the ellipse tangency condition in terms of $\vec{m}$.\\
Substitute $a=2\vec{e}_1^\top\vec{m},\; b=2\vec{e}_2^\top\vec{m}:$
\begin{align}
\frac{25}{4\brak{\vec{e}_1^\top\vec{m}}^2}+\frac{4}{4\brak{\vec{e}_2^\top\vec{m}}^2}=1.
\end{align}
\end{frame}

% ------------------ Next Slide ------------------
\begin{frame}{Solution}
\begin{align}
\implies \brak{4\brak{\vec{e}_1^\top\vec{m}}^2-25}\,\brak{\vec{e}_2^\top\vec{m}}^2
=4\brak{\vec{e}_1^\top\vec{m}}^2.
\end{align}

\begin{align}
\text{Or equivalently, }
4\brak{\vec{e}_1^\top\vec{m}}^2\brak{\vec{e}_2^\top\vec{m}}^2
-4\brak{\vec{e}_1^\top\vec{m}}^2
-25\brak{\vec{e}_2^\top\vec{m}}^2
=0.
\end{align}

\begin{align}
\text{Finally, let } 
\vec{m}=\myvec{x\\[4pt] y},
\quad \vec{e}_1^\top\vec{m}=x,\quad \vec{e}_2^\top\vec{m}=y.
\end{align}
\end{frame}

% ------------------ Next Slide ------------------
\begin{frame}{Solution}
\begin{align}
\text{Substituting gives the locus equation }
4x^2y^2 - 4x^2 - 25y^2 = 0.
\end{align}

\begin{align}
\text{Required locus: } 4x^2y^2 - 4x^2 - 25y^2 = 0.
\end{align}
\end{frame}

\begin{frame}{Graphical Representation}
\begin{figure}[H]
\begin{center}
\includegraphics[width=0.6\columnwidth]{figs/fig.png}
\end{center}
\caption{}
\label{fig:Fig1}
\end{figure}
\end{frame}

\begin{frame}{Codes}
\url{https://github.com/AarushDilawri/ee1030-2025/tree/main/ee25btech11001/MATGEO/10.7.50/codes}
\end{frame}
\end{document}


