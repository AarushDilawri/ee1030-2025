\documentclass{beamer}
\usepackage[utf8]{inputenc}

\usetheme{Madrid}
\usecolortheme{default}
\usepackage{amsmath,amssymb,amsfonts,amsthm}
\usepackage{txfonts}
\usepackage{tkz-euclide}
\usepackage{listings}
\usepackage{adjustbox}
\usepackage{array}
\usepackage{tabularx}
\usepackage{gvv}
\usepackage{lmodern}
\usepackage{circuitikz}
\usepackage{tikz}
\usepackage{graphicx}
\usepackage{lmodern}
\usepackage{enumitem}

\setbeamertemplate{page number in head/foot}[totalframenumber]

\title{5.13.27}
\date{\today}
\author{EE25BTECH11001 - Aarush Dilawri}

\begin{document}
\frame{\titlepage}

\begin{frame}{Matrix Commutation Problem}
\textbf{Question:} Let $\vec{A}=\myvec{1 & 2 \\ 3 & 4}$ and $\vec{B} = \myvec{a & 0 \\ 0 & b}, a,b \in \mathbb{N}$.

\begin{enumerate}[label=(\alph*)]
    \item there cannot exist any $\vec{B}$ such that $\vec{A}\vec{B} = \vec{B}\vec{A}$
    \item there exist more than one but finite number of $\vec{B}$ such that $\vec{A}\vec{B} = \vec{B}\vec{A}$
    \item there exists exactly one $\vec{B}$ such that $\vec{A}\vec{B} = \vec{B}\vec{A}$
    \item there exist infinitely many $\vec{B}$ such that $\vec{A}\vec{B} = \vec{B}\vec{A}$
\end{enumerate}
\end{frame}

\begin{frame}{Solution}
\textbf{Solution:} \\
\begin{align}
\text{Let } \vec{A} &= \myvec{1 & 2 \\ 3 & 4}, \quad 
\vec{B} = \myvec{a & 0 \\ 0 & b}, \quad a,b \in \mathbb{N}.
\end{align}
We compute $\vec{A}\vec{B}$:
\begin{align}
\vec{A}\vec{B} 
&= \myvec{1 & 2 \\ 3 & 4} \myvec{a & 0 \\ 0 & b} \\[6pt]
&= \myvec{a & 2b \\ 3a & 4b}.
\end{align}    
\end{frame}
\begin{frame}{Solution}
Similarly, compute $\vec{B}\vec{A}$:
\begin{align}
\vec{B}\vec{A} 
&= \myvec{a & 0 \\ 0 & b} \myvec{1 & 2 \\ 3 & 4} \\[6pt]
&= \myvec{a & 2a \\ 3b & 4b}.
\end{align}    
\end{frame}
\begin{frame}{Solution}
For $\vec{A}\vec{B} = \vec{B}\vec{A}$, we must have:
\begin{align}
\myvec{a & 2b \\ 3a & 4b} = \myvec{a & 2a \\ 3b & 4b}.
\end{align}
Equating the corresponding entries gives:
\begin{align}
2b &= 2a \quad \implies \quad b = a, \\[6pt]
3a &= 3b \quad \implies \quad a = b.
\end{align}    
\end{frame}
\begin{frame}{Solution}
Hence, 
\begin{align}
\vec{B} = \myvec{a & 0 \\ 0 & a} = a\vec{I}.
\end{align}
Since $a \in \mathbb{N}$, there are infinitely many such $\vec{B}$.\\
Therefore, the answer is (d) there exist infinitely many $\vec{B}$ such that $\vec{AB} = \vec{BA}$.    
\end{frame}
\begin{frame}[fragile]{C Code (code.c)}
\begin{lstlisting}[language=C]
#int countCommutingMatrices(int n, int *A, int maxVal) {
    // if any off-diagonal entry of A is non-zero, infinite solutions (a=b)
    for (int i = 0; i < n; i++)
        for (int j = 0; j < n; j++)
            if (i != j && A[i*n + j] != 0)
                return -1;

    // otherwise finite: count diagonal Bs with a=b
    return maxVal;
}
\end{lstlisting}
\end{frame}
\begin{frame}[fragile]{Python Code (code.py)}
\begin{lstlisting}[language=Python]
def count_commuting(A, n, maxVal):
    for i in range(n):
        for j in range(n):
            if i != j and A[i][j] != 0:
                return -1
    return maxVal

A = [[1, 2],
     [3, 4]]
n = 2
maxVal = 5

res = count_commuting(A, n, maxVal)
print("Number of commuting matrices B = ∞ (infinite)" if res == -1 else f"Number of B = {res}")
\end{lstlisting}
\end{frame}
\begin{frame}[fragile]{Python Code (nativecode.py)}
\begin{lstlisting}[language=Python]
import ctypes

lib = ctypes.CDLL("./code.so")
lib.countCommutingMatrices.argtypes = [ctypes.c_int, ctypes.POINTER(ctypes.c_int), ctypes.c_int]
lib.countCommutingMatrices.restype = ctypes.c_int

n = 2
A = [1, 2, 3, 4]
maxVal = 5
A_c = (ctypes.c_int * (n*n))(*A)

res = lib.countCommutingMatrices(n, A_c, maxVal)
print("Number of commuting matrices B = ∞ (infinite)" if res == -1 else f"Number of B = {res}")
\end{lstlisting}
\end{frame}
\end{document}