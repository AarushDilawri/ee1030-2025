\let\negmedspace\undefined
\let\negthickspace\undefined
\documentclass[journal]{IEEEtran}
\usepackage[a5paper, margin=10mm, onecolumn]{geometry}
%\usepackage{lmodern} % Ensure lmodern is loaded for pdflatex
\usepackage{tfrupee} % Include tfrupee package

\setlength{\headheight}{1cm} % Set the height of the header box
\setlength{\headsep}{0mm}     % Set the distance between the header box and the top of the text

\usepackage{gvv-book}
\usepackage{gvv}
\usepackage{cite}
\usepackage{amsmath,amssymb,amsfonts,amsthm}
\usepackage{algorithmic}
\usepackage{graphicx}
\usepackage{textcomp}
\usepackage{xcolor}
\usepackage{txfonts}
\usepackage{listings}
\usepackage{enumitem}
\usepackage{mathtools}
\usepackage{gensymb}
\usepackage{comment}
\usepackage{multicol}
\usepackage[breaklinks=true]{hyperref}
\usepackage{tkz-euclide} 
\usepackage{listings}
% \usepackage{gvv}                                        
\def\inputGnumericTable{}                                 
\usepackage[latin1]{inputenc}                                
\usepackage{color}                                            
\usepackage{array}                                            
\usepackage{longtable}                                       
\usepackage{calc}                                             
\usepackage{multirow}                                         
\usepackage{hhline}                                           
\usepackage{ifthen}                                           
\usepackage{lscape}
\usepackage{circuitikz}
\tikzstyle{block} = [rectangle, draw, fill=blue!20, 
    text width=4em, text centered, rounded corners, minimum height=3em]
\tikzstyle{sum} = [draw, fill=blue!10, circle, minimum size=1cm, node distance=1.5cm]
\tikzstyle{input} = [coordinate]
\tikzstyle{output} = [coordinate]


\begin{document}

\bibliographystyle{IEEEtran}
\vspace{3cm}

\title{5.13.27}
\author{EE25BTECH11001 - Aarush Dilawri}
\maketitle
% \newpage
% \bigskip
{\let\newpage\relax\maketitle}

\renewcommand{\thefigure}{\theenumi}
\renewcommand{\thetable}{\theenumi}
\setlength{\intextsep}{10pt} % Space between text and floats


\numberwithin{equation}{enumi}
\numberwithin{figure}{enumi}
\renewcommand{\thetable}{\theenumi}

\textbf{Question:} Let $\vec{A}=\myvec{1 & 2 \\ 3 & 4}$ and $\vec{B} = \myvec{a & 0 \\ 0 & b}, a,b \in \mathbb{N}$.


\begin{enumerate}[label=(\alph*)]
    \item there cannot exist any $\vec{B}$ such that $\vec{A}\vec{B} = \vec{B}\vec{A}$
    \item there exist more than one but finite number of $\vec{B}$ such that $\vec{A}\vec{B} = \vec{B}\vec{A}$
    \item there exists exactly one $\vec{B}$ such that $\vec{A}\vec{B} = \vec{B}\vec{A}$
    \item there exist infinitely many $\vec{B}$ such that $\vec{A}\vec{B} = \vec{B}\vec{A}$
\end{enumerate}


\solution \\
\begin{align}
\text{Let } \vec{A} &= \myvec{1 & 2 \\ 3 & 4}, \quad 
\vec{B} = \myvec{a & 0 \\ 0 & b}, \quad a,b \in \mathbb{N}.
\end{align}

We compute $\vec{A}\vec{B}$:
\begin{align}
\vec{A}\vec{B} 
&= \myvec{1 & 2 \\ 3 & 4} \myvec{a & 0 \\ 0 & b} \\[6pt]
&= \myvec{a & 2b \\ 3a & 4b}.
\end{align}

Similarly, compute $\vec{B}\vec{A}$:
\begin{align}
\vec{B}\vec{A} 
&= \myvec{a & 0 \\ 0 & b} \myvec{1 & 2 \\ 3 & 4} \\[6pt]
&= \myvec{a & 2a \\ 3b & 4b}.
\end{align}

For $\vec{A}\vec{B} = \vec{B}\vec{A}$, we must have:
\begin{align}
\myvec{a & 2b \\ 3a & 4b} = \myvec{a & 2a \\ 3b & 4b}.
\end{align}

Equating the corresponding entries gives:
\begin{align}
2b &= 2a \quad \implies \quad b = a, \\[6pt]
3a &= 3b \quad \implies \quad a = b.
\end{align}

Hence, 
\begin{align}
\vec{B} = \myvec{a & 0 \\ 0 & a} = a\vec{I}.
\end{align}

Since $a \in \mathbb{N}$, there are infinitely many such $\vec{B}$.\\
Therefore, the answer is (d) there exist infinitely many $\vec{B}$ such that $\vec{AB} = \vec{BA}$.
\end{document}